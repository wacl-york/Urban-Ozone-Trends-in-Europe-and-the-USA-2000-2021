%% Copernicus Publications Manuscript Preparation Template for LaTeX Submissions
%% ---------------------------------
%% This template should be used for copernicus.cls
%% The class file and some style files are bundled in the Copernicus Latex Package, which can be downloaded from the different journal webpages.
%% For further assistance please contact Copernicus Publications at: production@copernicus.org
%% https://publications.copernicus.org/for_authors/manuscript_preparation.html


%% Please use the following documentclass and journal abbreviations for preprints and final revised papers.

%% 2-column papers and preprints
\documentclass[journal abbreviation, manuscript]{copernicus}



%% Journal abbreviations (please use the same for preprints and final revised papers)


% Advances in Geosciences (adgeo)
% Advances in Radio Science (ars)
% Advances in Science and Research (asr)
% Advances in Statistical Climatology, Meteorology and Oceanography (ascmo)
% Aerosol Research (ar)
% Annales Geophysicae (angeo)
% Archives Animal Breeding (aab)
% Atmospheric Chemistry and Physics (acp)
% Atmospheric Measurement Techniques (amt)
% Biogeosciences (bg)
% Climate of the Past (cp)
% DEUQUA Special Publications (deuquasp)
% Earth Surface Dynamics (esurf)
% Earth System Dynamics (esd)
% Earth System Science Data (essd)
% E&G Quaternary Science Journal (egqsj)
% EGUsphere (egusphere) | This is only for EGUsphere preprints submitted without relation to an EGU journal.
% European Journal of Mineralogy (ejm)
% Fossil Record (fr)
% Geochronology (gchron)
% Geographica Helvetica (gh)
% Geoscience Communication (gc)
% Geoscientific Instrumentation, Methods and Data Systems (gi)
% Geoscientific Model Development (gmd)
% History of Geo- and Space Sciences (hgss)
% Hydrology and Earth System Sciences (hess)
% Journal of Bone and Joint Infection (jbji)
% Journal of Micropalaeontology (jm)
% Journal of Sensors and Sensor Systems (jsss)
% Magnetic Resonance (mr)
% Mechanical Sciences (ms)
% Natural Hazards and Earth System Sciences (nhess)
% Nonlinear Processes in Geophysics (npg)
% Ocean Science (os)
% Polarforschung - Journal of the German Society for Polar Research (polf)
% Primate Biology (pb)
% Proceedings of the International Association of Hydrological Sciences (piahs)
% Safety of Nuclear Waste Disposal (sand)
% Scientific Drilling (sd)
% SOIL (soil)
% Solid Earth (se)
% State of the Planet (sp)
% The Cryosphere (tc)
% Weather and Climate Dynamics (wcd)
% Web Ecology (we)
% Wind Energy Science (wes)


%% \usepackage commands included in the copernicus.cls:
%\usepackage[german, english]{babel}
%\usepackage{tabularx}
%\usepackage{cancel}
%\usepackage{multirow}
%\usepackage{supertabular}
%\usepackage{algorithmic}
%\usepackage{algorithm}
%\usepackage{amsthm}
%\usepackage{float}
%\usepackage{subfig}
%\usepackage{rotating}

\usepackage{booktabs} 
\usepackage{adjustbox}

\begin{document}

\title{Urban Ozone Trends in Europe and the USA (2000-2021)}


% \Author[affil]{given_name}{surname}

\Author[1,2,*][beth.nelson@york.ac.uk]{Beth S.}{Nelson} %% correspondence author
\Author[1,2,*][will.drysdale@york.ac.uk]{Will S.}{Drysdale}

\affil[1]{Wolfson Atmospheric Chemistry Laboratories, Department of Chemistry, University of York, Heslington, York, YO10 5DD, UK}
\affil[2]{National Centre for Atmospheric Science, University of York, York, UK}
\affil[*]{These authors contributed equally to this work.}
%% The [] brackets identify the author with the corresponding affiliation. 1, 2, 3, etc. should be inserted.

%% If an author is deceased, please mark the respective author name(s) with a dagger, e.g. "\Author[2,$\dag$]{Anton}{Smith}", and add a further "\affil[$\dag$]{deceased, 1 July 2019}".

%% If authors contributed equally, please mark the respective author names with an asterisk, e.g. "\Author[2,*]{Anton}{Smith}" and "\Author[3,*]{Bradley}{Miller}" and add a further affiliation: "\affil[*]{These authors contributed equally to this work.}".


\runningtitle{TEXT}

\runningauthor{TEXT}





\received{}
\pubdiscuss{} %% only important for two-stage journals
\revised{}
\accepted{}
\published{}

%% These dates will be inserted by Copernicus Publications during the typesetting process.


\firstpage{1}

\maketitle



\begin{abstract}
TEXT
\end{abstract}


\copyrightstatement{TEXT} %% This section is optional and can be used for copyright transfers.


\section{Introduction}  %% \introduction[modified heading if necessary]
Tropospheric ozone is a greenhouse gas and air pollutant harmful to human health, and plant growth (Fleming et al., 2018; Mills et al., 2018; Szopa et al., 2021). It is a secondary air pollutant, formed from the photochemical reactions of primary pollutants NOx (NO + NO2) and volatile organic compounds (VOCs). Despite global successes in reducing primary pollutant emissions over the past few decades, global exposure to ozone has been increasing throughout the 21st century. This is particularly observed in urban areas, where the vast majority of the global population live, projected to increase to 68\% in 2050 from 55\% in 2018 (UN, 2019). In a study of 12946 cities located worldwide, the average mean weighted ozone concentration increased by 11\% between 2000 and 2019, and the number of cities exceeding the WHO peak season ozone standard increased from 89\% to 96\% (Malashock et al., 2022).

Due to the complexity of ozone production, its trend direction, magnitude, and significance varies by location. A previous study calculated trends in two ozone metrics globally over the period of 2000 - 2014: 4th highest daily maximum 8-hour ozone (4MDA8), and the number of days with MDA8 > 70 ppb ozone (NDGT70) (Fleming et al., 2018). The study used data from 4801 global monitoring sites over this time period. For both of these metrics, downward trends were observed for most of the USA, and some sites in Europe. However, over the period of 2010-2014 (2,600 sites utilised), sites located in regions with the highest ozone precursor emissions across North America, Europe and East Asia had the highest values in 4MDA8 and NDGT70. In North America and Europe, this was particularly true for California and parts of southern Europe (Fleming et al., 2018).

Since the 1990s, a general downward trend in urban ozone pollution has been observed in the United States (He et al., 2020). This reducing trend has been linked to stricter limiting regulations on the emissions of primary pollutants such as NOx and VOC. Although NOx and VOC emissions in Europe have also been declining since the late 1980s, the trend in ozone is less clear due to large inner-annual variation, driven by climate variability and the dispersion and transport of pollutants from other regions (Jonson et al., 2006; Yan et al., 2018). Between 1995 and 2014, negative trends in the highest ozone levels across urban sites in Europe were identified due to pollutant emission restrictions across Europe. However, increasing background levels, particularly in northern and eastern Europe, make it difficult to identify strong trends in urban ozone when transboundary effects are considered. (Yan et al., 2018). Despite this, a study of 93 suburban and urban sites across Europe identified notable enhancements in ozone seasonal and annual means between 1995 - 2012, even with the continuous downward trend in anthropogenic emissions across the continent (Yan et al., 2018).

Since these earlier studies, much of the literature focuses on the impact of the COVID-19 pandemic on air pollutant concentrations and trends (Lee et al., 2020; Shi et al., 2021; Grange et al., 2021). A study of eleven cities across the world, all of which implemented stringent lockdown measures in early 2020, captured sudden decreases in deweathered NO2 concentrations, concurrent with sudden increases in O3, in most locations. (Shi et al., 2021). Another study employing machine learning models to predict business-as-usual levels of pollutants, estimated that NO2 concentrations were 32\% lower than expected across 102 European urban background locations, whereas O3 was 21\% higher (Grange et al., 2021). It is highly likely that this global event has resulted in perturbations of both long-term NO2 and O3 trends, particularly in locations where lockdowns were stringent or lengthy.



\section{Methodology}
Hourly NO2 and O3 data were obtained for urban sites in the USA and Europe using the TOAR-II (REF) and European Environment Agency (EEA) (REF) databases, using the r-packages toarR (REF) and saqgetr (REF) respectively. TOAR-II data was retrieved between 2000-01-01 and 2021-12-31 (the latest available) and EEA data between 2000-01-01 and 2023-12-31, the latter being extended longer so the effects of the COVID-19 pandemic could be observed more clearly. Time series were required to have 90 \% data coverage between 2000-01-01 and 2021-12-31 and retained time series were averaged to daily medians. Additionally, a small number of time series were removed following visual inspection - these are listed in table SI.XX. This resulted in 228 O3 time series (144 Europe, 84 USA), 323 NO2 time series (246 Europe, 77 USA) and 126 Ox (NO2 + O3) time series (101 Europe, 25 USA).

Firstly, the time series were de-seasoned by calculating a monthly median climatology and subtracting this from the timeseries, producing an ‘anomaly’ time series. The NO2 and O3 anomaly time series were summed to produce Ox anomaly time series for sites that have both NO2 and O3 data available. Following this, several methods were defined to help describe trends. Locally Estimated Scatterplot Smoothing (LOESS) was applied, with a smoothing parameter ($\alpha$) chosen as 0.5, which was determined by inspection to capture a balance of broad trends and medium term non-linearity. LOESS works well for drawing the eye to features of the time series, but does not allow for broad trends to be described, as each point is described by a different regression. As such, quantile regression (QR) was used to define trends with the code for running these being adapted from Chang et al. 2023. Quantile regression calculates a linear model that seeks to minimise the residuals with a defined proportion ($\tau$) of the points above and below the fit line. For example the scenario $\tau$ = 0.5 splits the data 50:50 above and below the line. QR has the advantage of being insensitive to outliers, and can be considered analogous to a “median” trend line. This is desirable for the longer term trends being investigated here. However, a single QR is the other extreme compared to LOESS - using a single slope to define the trend across a complex time series. As such piecewise quantile regressions (PQR) with up to two break points were calculated, with the goal of allowing some changes in the trends calculated, while still being able to describe them with a small number of coefficients.

To determine what breakpoints to use on each time series, 109 scenarios per time series defined by 1, 2 or 3 QRs were created, with break points occurring on the 1st of January 2002 - 2022 and in the cases where there were 2 breakpoints, these could not occur within 5 years of each other. The cases with 2 or 3 regressions were combined into a single PQR, (PQR1, PQR2). All regressions were calculated at $\tau$ = 0.05, 0.10, 0.25, 0.50, 0.75, 0.95 and 0.95, and did not vary between segments of a piecewise regression.

To select ‘change points’ from the array of breakpoints available, the model performance was evaluated via Akaike information criterion (AIC). The piecewise scenario with the minimum AIC at $\tau$ = 0.5 were selected as the change points for a given time series. It is possible that the change points would vary with $\tau$, however, it was decided to only determine these at $\tau$ = 0.5 to allow comparison of the same change points across the various $\tau$ values. It would be increasingly difficult to draw conclusions if both $\tau$ and change points could vary within a time series.  Figure XX demonstrates this process on a single timeseries. Change points could be assessed for their efficacy (i.e. is the effect of introducing a change point important) by comparing the QR and the best PQR1 and PQR2 cases, but as this does not substantively change the results presented here, this was not done and the PQR2 fits were determined to be the most appropriate in all cases.


\section{Results and Discussion}
The dataset that results from the application of this method has several degrees of freedom to bear in mind during discussion. For a given timeseries (one species at a one site) there are two break points that are common for all the trend lines, and as there are 7 quantiles that the slopes have been calculated for this results in 21 slopes, each with an associated p-value. When comparing between sites (or even species within a site) the breakpoints can differ. This is necessary to capture the changing trends observed at urban sites but does complicate describing them. For this discussion we have grouped the sites into those from Europe and those from the United States, though details on individual European countries are available in the SI. 
To aid in summarising, in some cases the results have been grouped into time periods of a few years – in these cases if a trend changes due to a break point both trends have been counted. For example, if between 2000 and 2004 a site were to go from increasing O3 to decreasing O3, this would be described as ‘between 2000 and 2004 there was one increasing O3 trend and one decreasing O3 trend’. In the case where a site had a break point in 2003, but the trend remained increasing, this will be counted as one increasing trend. This has the benefit that no one category can count more than then number of available time series. In visualisation, both trends would be shown.
Slopes were the p-value is > 0.33 have been classed as ‘No Trend’ regardless of the magnitude (generally we observer that as the magnitude of the trend decreases so does its significance).

\subsection{Overview of Trends}
To begin with, the $\tau$ = 0.5 trends were examined and are presented on maps in figures AA, BB, CC and DD. 

In Europe, between 2000 and 2004 there were 108 trends where O3 was increasing, and 41 where it was decreasing, 32 showed no trend. By 2015-2021 123 had an increasing trend, 32 decreasing and 41 with no trend, indicating generally increasing O3 concentrations across Europe. This is somewhat mirrored by the trends in NO2, where in 2000-2004 69 trends where increasing, 182 decreasing and 59 with no trend, changing to 30 increasing, 241 decreasing and 39 with no trend by 2015-2021. This is commensurate with the general picture that urban locations in Europe are still on the VOC limited portion of the O3 production isopleth, and decreasing NOx increases O3 production. 
This contrasts with the USA, were in 2000-2004, there were 10 sites with increasing NO2, but 0 sites increasing between 2005 and 2014, but in 2015 -2021 17 sites had begun increasing in NO2 again – though notably as seen from figure CC these are different sites to those that were increasing at the beginning of the century. This increase in NO2 is not clearly reflected by the O3 trends with 62 / 23 / 22 increasing / decreasing / no trend in 2000 – 2004 and 44 / 34 / 25 increasing / decreasing / no trend, with a decreasing number of sites O3 with a positive trend in O3, suggesting that the majority of decreasing NO2 trends are slowing the increase in O3. The counts of sites at all values of $\tau$ can be found in tables XX and YY. 
For O3 in Europe, at sites with increasing trends the median slope was 0.36 ppbv year-1, slightly lower than in the US, where they were increasing by 0.40 ppbv year-1. For NO2, the median rate of increase was the same in Europe 0.36 ppbv year-1, but higher at 0.71 ppbv year-1 in the US. For decreasing trends, O3 and NO2 in Europe and the US were all similar, O3 in Europe decreasing at -0.40 ppb year-1, (-0.36 ppbv year-1) in the US and for NO2 -0.39 ppbv year-1 in Europe and -0.47 in the US. This information along with the 5 th and 95 th percentile (still at $\tau$ = 0.5) are shown in table ZZ. 


\input{tables/slope_ranges.txt}

\input{tables/europe_segs_11_14.txt}

\input{tables/usa_segs_11_14.txt}



\subsection{HEADING}
TEXT


\subsubsection{HEADING}
TEXT


\section{Conclusion}  %% \conclusions[modified heading if necessary]
TEXT

%% The following commands are for the statements about the availability of data sets and/or software code corresponding to the manuscript.
%% It is strongly recommended to make use of these sections in case data sets and/or software code have been part of your research the article is based on.

\codeavailability{TEXT} %% use this section when having only software code available


\dataavailability{TEXT} %% use this section when having only data sets available


\codedataavailability{TEXT} %% use this section when having data sets and software code available


\sampleavailability{TEXT} %% use this section when having geoscientific samples available


\videosupplement{TEXT} %% use this section when having video supplements available


\appendix
\section{}    %% Appendix A

\subsection{}     %% Appendix A1, A2, etc.


\noappendix       %% use this to mark the end of the appendix section. Otherwise the figures might be numbered incorrectly (e.g. 10 instead of 1).

%% Regarding figures and tables in appendices, the following two options are possible depending on your general handling of figures and tables in the manuscript environment:

%% Option 1: If you sorted all figures and tables into the sections of the text, please also sort the appendix figures and appendix tables into the respective appendix sections.
%% They will be correctly named automatically.

%% Option 2: If you put all figures after the reference list, please insert appendix tables and figures after the normal tables and figures.
%% To rename them correctly to A1, A2, etc., please add the following commands in front of them:

\appendixfigures  %% needs to be added in front of appendix figures

\appendixtables   %% needs to be added in front of appendix tables

%% Please add \clearpage between each table and/or figure. Further guidelines on figures and tables can be found below.



\authorcontribution{TEXT} %% this section is mandatory

\competinginterests{TEXT} %% this section is mandatory even if you declare that no competing interests are present

\disclaimer{TEXT} %% optional section

\begin{acknowledgements}
TEXT
\end{acknowledgements}




%% REFERENCES

%% The reference list is compiled as follows:

\begin{thebibliography}{}

\bibitem[AUTHOR(YEAR)]{LABEL1}
REFERENCE 1

\bibitem[AUTHOR(YEAR)]{LABEL2}
REFERENCE 2

\end{thebibliography}

%% Since the Copernicus LaTeX package includes the BibTeX style file copernicus.bst,
%% authors experienced with BibTeX only have to include the following two lines:
%%
%% \bibliographystyle{copernicus}
%% \bibliography{example.bib}
%%
%% URLs and DOIs can be entered in your BibTeX file as:
%%
%% URL = {http://www.xyz.org/~jones/idx_g.htm}
%% DOI = {10.5194/xyz}


%% LITERATURE CITATIONS
%%
%% command                        & example result
%% \citet{jones90}|               & Jones et al. (1990)
%% \citep{jones90}|               & (Jones et al., 1990)
%% \citep{jones90,jones93}|       & (Jones et al., 1990, 1993)
%% \citep[p.~32]{jones90}|        & (Jones et al., 1990, p.~32)
%% \citep[e.g.,][]{jones90}|      & (e.g., Jones et al., 1990)
%% \citep[e.g.,][p.~32]{jones90}| & (e.g., Jones et al., 1990, p.~32)
%% \citeauthor{jones90}|          & Jones et al.
%% \citeyear{jones90}|            & 1990



%% FIGURES

%% When figures and tables are placed at the end of the MS (article in one-column style), please add \clearpage
%% between bibliography and first table and/or figure as well as between each table and/or figure.

% The figure files should be labelled correctly with Arabic numerals (e.g. fig01.jpg, fig02.png).


%% ONE-COLUMN FIGURES

%%f
%\begin{figure}[t]
%\includegraphics[width=8.3cm]{FILE NAME}
%\caption{TEXT}
%\end{figure}
%
%%% TWO-COLUMN FIGURES
%
%%f
%\begin{figure*}[t]
%\includegraphics[width=12cm]{FILE NAME}
%\caption{TEXT}
%\end{figure*}
%
%
%%% TABLES
%%%
%%% The different columns must be seperated with a & command and should
%%% end with \\ to identify the column brake.
%
%%% ONE-COLUMN TABLE
%
%%t
%\begin{table}[t]
%\caption{TEXT}
%\begin{tabular}{column = lcr}
%\tophline
%
%\middlehline
%
%\bottomhline
%\end{tabular}
%\belowtable{} % Table Footnotes
%\end{table}
%
%%% TWO-COLUMN TABLE
%
%%t
%\begin{table*}[t]
%\caption{TEXT}
%\begin{tabular}{column = lcr}
%\tophline
%
%\middlehline
%
%\bottomhline
%\end{tabular}
%\belowtable{} % Table Footnotes
%\end{table*}
%
%%% LANDSCAPE TABLE
%
%%t
%\begin{sidewaystable*}[t]
%\caption{TEXT}
%\begin{tabular}{column = lcr}
%\tophline
%
%\middlehline
%
%\bottomhline
%\end{tabular}
%\belowtable{} % Table Footnotes
%\end{sidewaystable*}
%
%
%%% MATHEMATICAL EXPRESSIONS
%
%%% All papers typeset by Copernicus Publications follow the math typesetting regulations
%%% given by the IUPAC Green Book (IUPAC: Quantities, Units and Symbols in Physical Chemistry,
%%% 2nd Edn., Blackwell Science, available at: http://old.iupac.org/publications/books/gbook/green_book_2ed.pdf, 1993).
%%%
%%% Physical quantities/variables are typeset in italic font (t for time, T for Temperature)
%%% Indices which are not defined are typeset in italic font (x, y, z, a, b, c)
%%% Items/objects which are defined are typeset in roman font (Car A, Car B)
%%% Descriptions/specifications which are defined by itself are typeset in roman font (abs, rel, ref, tot, net, ice)
%%% Abbreviations from 2 letters are typeset in roman font (RH, LAI)
%%% Vectors are identified in bold italic font using \vec{x}
%%% Matrices are identified in bold roman font
%%% Multiplication signs are typeset using the LaTeX commands \times (for vector products, grids, and exponential notations) or \cdot
%%% The character * should not be applied as mutliplication sign
%
%
%%% EQUATIONS
%
%%% Single-row equation
%
%\begin{equation}
%
%\end{equation}
%
%%% Multiline equation
%
%\begin{align}
%& 3 + 5 = 8\\
%& 3 + 5 = 8\\
%& 3 + 5 = 8
%\end{align}
%
%
%%% MATRICES
%
%\begin{matrix}
%x & y & z\\
%x & y & z\\
%x & y & z\\
%\end{matrix}
%
%
%%% ALGORITHM
%
%\begin{algorithm}
%\caption{...}
%\label{a1}
%\begin{algorithmic}
%...
%\end{algorithmic}
%\end{algorithm}
%
%
%%% CHEMICAL FORMULAS AND REACTIONS
%
%%% For formulas embedded in the text, please use \chem{}
%
%%% The reaction environment creates labels including the letter R, i.e. (R1), (R2), etc.
%
%\begin{reaction}
%%% \rightarrow should be used for normal (one-way) chemical reactions
%%% \rightleftharpoons should be used for equilibria
%%% \leftrightarrow should be used for resonance structures
%\end{reaction}
%
%
%%% PHYSICAL UNITS
%%%
%%% Please use \unit{} and apply the exponential notation


\end{document}
