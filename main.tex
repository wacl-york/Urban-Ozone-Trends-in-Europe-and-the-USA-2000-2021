%% Copernicus Publications Manuscript Preparation Template for LaTeX Submissions
%% ---------------------------------
%% This template should be used for copernicus.cls
%% The class file and some style files are bundled in the Copernicus Latex Package, which can be downloaded from the different journal webpages.
%% For further assistance please contact Copernicus Publications at: production@copernicus.org
%% https://publications.copernicus.org/for_authors/manuscript_preparation.html


%% Please use the following documentclass and journal abbreviations for preprints and final revised papers.

%% 2-column papers and preprints
\documentclass[journal abbreviation, manuscript]{copernicus}



%% Journal abbreviations (please use the same for preprints and final revised papers)


% Advances in Geosciences (adgeo)
% Advances in Radio Science (ars)
% Advances in Science and Research (asr)
% Advances in Statistical Climatology, Meteorology and Oceanography (ascmo)
% Aerosol Research (ar)
% Annales Geophysicae (angeo)
% Archives Animal Breeding (aab)
% Atmospheric Chemistry and Physics (acp)
% Atmospheric Measurement Techniques (amt)
% Biogeosciences (bg)
% Climate of the Past (cp)
% DEUQUA Special Publications (deuquasp)
% Earth Surface Dynamics (esurf)
% Earth System Dynamics (esd)
% Earth System Science Data (essd)
% E&G Quaternary Science Journal (egqsj)
% EGUsphere (egusphere) | This is only for EGUsphere preprints submitted without relation to an EGU journal.
% European Journal of Mineralogy (ejm)
% Fossil Record (fr)
% Geochronology (gchron)
% Geographica Helvetica (gh)
% Geoscience Communication (gc)
% Geoscientific Instrumentation, Methods and Data Systems (gi)
% Geoscientific Model Development (gmd)
% History of Geo- and Space Sciences (hgss)
% Hydrology and Earth System Sciences (hess)
% Journal of Bone and Joint Infection (jbji)
% Journal of Micropalaeontology (jm)
% Journal of Sensors and Sensor Systems (jsss)
% Magnetic Resonance (mr)
% Mechanical Sciences (ms)
% Natural Hazards and Earth System Sciences (nhess)
% Nonlinear Processes in Geophysics (npg)
% Ocean Science (os)
% Polarforschung - Journal of the German Society for Polar Research (polf)
% Primate Biology (pb)
% Proceedings of the International Association of Hydrological Sciences (piahs)
% Safety of Nuclear Waste Disposal (sand)
% Scientific Drilling (sd)
% SOIL (soil)
% Solid Earth (se)
% State of the Planet (sp)
% The Cryosphere (tc)
% Weather and Climate Dynamics (wcd)
% Web Ecology (we)
% Wind Energy Science (wes)


%% \usepackage commands included in the copernicus.cls:
%\usepackage[german, english]{babel}
%\usepackage{tabularx}
%\usepackage{cancel}
%\usepackage{multirow}
%\usepackage{supertabular}
%\usepackage{algorithmic}
%\usepackage{algorithm}
%\usepackage{amsthm}
%\usepackage{float}
%\usepackage{subfig}
%\usepackage{rotating}

\usepackage{booktabs} 
\usepackage{adjustbox}
\usepackage{longtable}

\begin{document}

\title{Urban Ozone Trends in Europe and the USA (2000-2021)}

% \Author[affil]{given_name}{surname}

\Author[1,2,*][beth.nelson@york.ac.uk]{Beth S.}{Nelson} %% correspondence author
\Author[1,2,*][will.drysdale@york.ac.uk]{Will S.}{Drysdale}

\affil[1]{Wolfson Atmospheric Chemistry Laboratories, Department of Chemistry, University of York, Heslington, York, YO10 5DD, UK}
\affil[2]{National Centre for Atmospheric Science, University of York, York, UK}
\affil[*]{These authors contributed equally to this work.}
%% The [] brackets identify the author with the corresponding affiliation. 1, 2, 3, etc. should be inserted.

%% If an author is deceased, please mark the respective author name(s) with a dagger, e.g. "\Author[2,$\dag$]{Anton}{Smith}", and add a further "\affil[$\dag$]{deceased, 1 July 2019}".

%% If authors contributed equally, please mark the respective author names with an asterisk, e.g. "\Author[2,*]{Anton}{Smith}" and "\Author[3,*]{Bradley}{Miller}" and add a further affiliation: "\affil[*]{These authors contributed equally to this work.}".


\runningtitle{TEXT}

\runningauthor{TEXT}

\received{}
\pubdiscuss{} %% only important for two-stage journals
\revised{}
\accepted{}
\published{}

%% These dates will be inserted by Copernicus Publications during the typesetting process.


\firstpage{1}

\maketitle


\begin{abstract}
Trends in urban O\textsubscript{3} and NO\textsubscript{2} across Europe and the United States of America were explored between 2000-2021. Using surface monitoring site data from the TOAR-II and European Environment Agency databases, piecewise quantile regression (PQR) analysis was performed on 228 O\textsubscript{3} time series (144 European, 84 USA) and 322 NO\textsubscript{2} times series (245 European, 77 USA). The PQR analysis permitted 2 break points over the 23 year period to balance the intent to describe changes over a large time period, while still capturing the abrupt changes that can occur in urban atmospheres. Regressions were performed over quantiles ranging from 0.05 to 0.95 and indications of a slowing in the increase of high European O\textsubscript{3} levels was observed. In Europe, more trends were found to having an increasing O\textsubscript{3} trend between 2015-2021 compared with 2000-2004. The reverse was true in the USA, with a reduction in the number of sites with increasing O\textsubscript{3} trends when the same periods were compared. An analysis of the change points revealed a large proportion of sites in Europe, were the second change point in NO\textsubscript{2} switched from a positive to negative trend, occurred in 2020 (41/43 second change points in this year). This was attributed a reduction in NO\textsubscript{2} due to the COVID-19 pandemic, however, in some cases these increasing trends have sustained beyond the recovery from restrictions. 
\end{abstract}

\section{Introduction}  %% \introduction[modified heading if necessary]
Tropospheric ozone (O\textsubscript{3}) is a greenhouse gas and air pollutant harmful to human health, and plant growth \citep{fleming_2018, mills_2018, szopa_2021}. It is a secondary air pollutant, formed from the photochemical reactions of primary pollutants NO\textsubscript{x} (NO + NO\textsubscript{2}) and volatile organic compounds (VOCs). The chemistry of O\textsubscript{3} formation is non-linear and the effect of changing precursor concentrations can cause O\textsubscript{3} concentrations to increase or decrease depending on the photochemical regime found at a given location \citep{ https://doi.org/10.1029/JD095iD02p01837, SILLMAN2002339}. Despite global successes in reducing primary pollutant emissions over the past few decades, global exposure to O\textsubscript{3} has been increasing throughout the 21st century. This is particularly observed in urban areas, where the vast majority of the global population live, projected to increase to 68\% in 2050 from 55\% in 2018 \citep{un_2019}. In a study of 12946 cities located worldwide, the average mean weighted O\textsubscript{3} concentration increased by 11\% between 2000 and 2019, and the number of cities exceeding the WHO peak season O\textsubscript{3} standard increased from 89\% to 96\% \citep{Malashock_2022}.

Due to the complexity of O\textsubscript{3} production, its trend direction, magnitude, and significance varies by location. A previous study calculated trends in two O\textsubscript{3} metrics globally over the period of 2000 - 2014: 4th highest daily maximum 8-hour O\textsubscript{3} (4MDA8), and the number of days with MDA8 > 70 ppb O\textsubscript{3} (NDGT70) \citep{fleming_2018}. The study used data from 4801 global monitoring sites over this time period. For both of these metrics, downward trends were observed for most of the USA, and some sites in Europe. However, over the period of 2010-2014 (2,600 sites utilised), sites located in regions with the highest O\textsubscript{3} precursor emissions across North America, Europe and East Asia had the highest values in 4MDA8 and NDGT70. In North America and Europe, this was particularly true for California and parts of southern Europe \citep{fleming_2018}.

Since the 1990s, a general downward trend in urban O\textsubscript{3} pollution has been observed in the United States \citep{acp-20-3191-2020}. This reducing trend has been linked to stricter limiting regulations on the emissions of primary pollutants such as NO\textsubscript{x} and VOCs. Although NO\textsubscript{x} and VOC emissions in Europe have also been declining since the late 1980s, the trend in O\textsubscript{3} is less clear due to large inner-annual variation, driven by climate variability and the dispersion and transport of pollutants from other regions \citep{acp-6-51-2006, acp-18-5589-2018}. Between 1995 and 2014, negative trends in the highest O\textsubscript{3} levels across urban sites in Europe were identified due to pollutant emission restrictions across Europe. However, increasing background levels, particularly in northern and eastern Europe, make it difficult to identify strong trends in urban O\textsubscript{3} when transboundary effects are considered \citep{acp-18-5589-2018}. Despite this, a study of 93 suburban and urban sites across Europe identified notable enhancements in O\textsubscript{3} seasonal and annual means between 1995 - 2012, even with the continuous downward trend in anthropogenic emissions across the continent \citep{acp-18-5589-2018}.

Since these earlier studies, much of the literature focuses on the impact of the COVID-19 pandemic on air pollutant concentrations and trends \citep{acp-20-15743-2020, doi:10.1126/sciadv.abd6696, acp-21-4169-2021, SOKHI2021106818}. A study of eleven cities across the world, all of which implemented stringent lockdown measures in early 2020, captured sudden decreases in deweathered NO\textsubscript{2} concentrations, concurrent with sudden increases in O\textsubscript{3}, in most locations \citep{doi:10.1126/sciadv.abd6696}. Another study employing machine learning models to predict business-as-usual levels of pollutants, estimated that NO\textsubscript{2} concentrations were 32\% lower than expected across 102 European urban background locations, whereas O\textsubscript{3} was 21\% higher \citep{acp-21-4169-2021}. It is highly likely that this global event has resulted in perturbations of both long-term NO\textsubscript{2} and O\textsubscript{3} trends, particularly in locations where lockdowns were stringent or lengthy.

This study examines the trends in urban O\textsubscript{3} and NO\textsubscript{2} using monitoring site data from 2020 - 2023. It employs quantile regression and change point detection to construct trends that capture the broad structure of a complex time series, while remaining explainable with concise statistics. Section \ref{sect:method} details the method used to create the trends, section \ref{sect:overview_of_trends} provides a high-level overview of the trends with section \ref{sect:significance_of_trends} focusing on the significance of the trends and highlighting features that are discussed in detail later on. Section \ref{sect:trends_across_quantiles} explores the information that can be gained from trends calculated over a range of quantiles and finally section \ref{sect:case_studies} provides details on some of the features revealed earlier in the analysis. 

\section{Methodology} \label{sect:method}

\subsection{Data Preparation} \label{sect:data_prep}
Hourly NO\textsubscript{2} and O\textsubscript{3} data were obtained for urban sites in the USA and Europe using the TOAR-II \citep{toar_db} and European Environment Agency (EEA) \citep{eea_1, eea_2} databases, using the r-packages \emph{toarR} \citep{drysdale_2024_14537446} and \emph{saqgetr} \citep{saqgetr} respectively. Those obtained from the EEA are categorised as urban traffic, urban background and urban industrial, while those from the TOAR database are categorised as urban and suburban. A full list of sites can be found in the supplementary information (table S1).  TOAR-II data was retrieved between 2000-01-01 and 2021-12-31 (the latest available) \textcolor{red}{and EEA data between 2000-01-01 and 2023-12-31, the latter being extended longer so the effects of the COVID-19 pandemic could be observed more clearly.} Time series were quality controlled via threshold, persistence and variance checks similar to those in \textcolor{red}{Wang et al., 2024 - or reference within}: O\textsubscript{3} concentrations were required be between 0 and 500 ppbv, periods where an identical value was repeated 8 or more times in a row were removed as were days where the difference between the minimum and maximum concentration was $\leq$ 2 ppbv. 

After quality control, time series were required to have 80 \% data coverage between 2000-01-01 and 2021-12-31 and for the calculation of annual metrics each individual year was retained if it had more than 60 \% coverage. Additionally, a small number of time series were removed following visual inspection - these are listed in table S2. This resulted in 360 O\textsubscript{3} time series (208 Europe, 152 USA). Trends were calculated on daily averaged data, either the mean or maximum daily 8-hour average (MDA8). 

For the mean data, subgroups of day (hours between 0800 - 1900 L inclusive) and night (hours between 2000 - 0700 L inclusive) were created, and for both mean and MDA8 data subgroups of warm (April - September) and cold (October - March) season were created (table \ref{tab:ts_types}). It is common to remove the seasonal component of an ozone time series to improve the accuracy of trends \citep{cooper_2020} and here we subtract monthly mean climatologies for each time series resulting ozone anomalies. 

\begin{table}[h]
\caption{Types of time series for which trends have been calculated \textcolor{red}{Could go in SI}}
\begin{tabular}{c|c c c}
Type              & Averaging & Hours         & Months            \\ \hline
Daily             & mean      & All           & All               \\
Daily Day         & mean      & 0800 - 1900 L & All               \\
Daily Day Warm    & mean      & 0800 - 1900 L & April - September \\
Daily Day Cold    & mean      & 0800 - 1900 L & October - March   \\
Daily Night       & mean      & 2000 - 0700 L & All               \\
Daily Night Warm  & mean      & 2000 - 0700 L & April - September \\
Daily Night Cold  & mean      & 2000 - 0700 L & October - March   \\
MDA8              & MDA8      & All           & All               \\
MDA8 Warm         & MDA8      & All           & April - September \\
MDA8 Cold         & MDA8      & All           & October - March   \\
\end{tabular}
\label{tab:ts_types}
\end{table}

\subsection{Metrics}

\subsection{Trend Analysis}
Trends were estimated for these daily, daily (daytime), daily (nighttime) and MDA8 series in three cases: all data, warm season (April - September) and cold season (October - March). These were calculated following the methodology in and using code provided by  \cite{chang2023guidancenotebeststatistical}. QR calculates a linear model that seeks to minimise the residuals with a defined proportion ($\tau$) of the points above and below the fit line. For example, the scenario $\tau$ = 0.5 splits the data 50:50 above and below the line. QR has the advantage of being insensitive to outliers and the $\tau$ = 0.5 case can be considered analogous to a “median” trend line. This is desirable for the longer-term trends being investigated here. The 1-sigma uncertainly for the QRs were calculated via a moving block bootstrapping method, where the data are subdivided into overlapping blocks that are $n^{1/4}$ points wide (or \textasciitilde{20} points for 20 years of hourly data). These blocks are shuffled to generate a new time series, and replicated 1000 times, and the standard deviation of these replicates calculated. This uncertainty was subsequently used in the determination of the p-value, providing a metric for the significance of the trends. 

Quantile regressions were calculated piecewise with 0 - 2 breakpoints to allow the trends to capture large non-linearities. As urban concentration trends can change sharply (e.g. with policy intervention), this gives the model the freedom to represent these. Limiting the model to a maximum of two breakpoints strikes a balance between capturing sufficiently large-scale changes while still being able to describe a \textasciitilde{20} year time series with a small number of coefficients.

To determine what break points to use on each time series, a range of candidate models were constructed, each with zero, one or two break points. These break points were restricted to be unable to occur within the first or last two years of the time series, nor could they occur within 5 years of each other. They were arbitrarily set to occur on the 1st January in a given year, which was deemed an appropriate degree of freedom given the 20 year span of the time series and locating the break points sub annually does not have a large effect on the resulting trends. For a time series that fully spanned the range of 2000-01-01 - 2021-12-31, 110 models would be created - one with zero break points, 18 with one break point, and 91 with 2. All regressions were calculated at $\tau$ = 0.05, 0.10, 0.25, 0.50, 0.75, 0.95 and 0.95.

To select 'change points' from the array of break points available, the model performance was evaluated via Akaike information criterion (AIC). AIC was chosen as the evaluation criteria oweing to its penalty term penalising models with more break points that do not appreciably improve the model fit over the simpler case, The model with the minimum AIC was selected from the candidate pool as the best fit for a time series at a given $\tau$. 

Trends have been collated into significance categories: p $\le$ 0.05 (high certainty), 0.05 $<$ p $\le$ 0.10 (medium certainty), 0.10 $<$ p $\le$ 0.33 (low certainty) and p > 0.33 (very low certainty or no evidence) based off of the guidance of \cite{chang2023guidancenotebeststatistical}. For the most part slopes where the p-value is > 0.33 are treated as ‘no trend’ regardless of their magnitude (generally we observe that as the magnitude of the trend decreases so does its significance), though sometimes are given with a direction when required by a visualisation.

\subsection{Clustering}
To aid analysis it was desirable to group similar time series, clusters were calculated using hierarchical clustering (HC) with dynamic time warping (DTW) used as a distance measure \citep{AGHABOZORGI201516}. DTW provides a distance measure between two time series that allows for some deviation in the relation of features in time and one-to-many mapping of features, as opposed to use of e.g. the euclidean distance where mapping between series is one-to-one and cannot undergo deformation \citep{Berndt_dtw, Abdullah_Keogh_dtw}. We adopt a similar method to \cite{REED2025110686} using the R package \emph{dtwclust} \citep{dtwclust} for both DTW and HC steps, and used to calculate cluster validity indices (CVIs). Clustering was performed by region and time series type (e.g sites in Europe for the MDA8-O3 warm season). Time series were normalised and then, to provide clusters related to the range of $\tau$ values used in the QRs, aggregated monthly by calculating the $\tau^{th}$ quantile per month. Metrics were not aggregated as they are all annual values. Clustering was calculated with 5 - 75 clusters, and then evaluated with 6 internal CVIs (Silhouette, Dunn, Calinski-Harabasz, COP, Davies-Bouldin and Modified Davies-Bouldin) where the latter 3 were inverted such that all can be maximised. The final selection for the number of clusters was determined as the mean of the optimal number of clusters suggested by each of the CVIs. 

\clearpage
\section{Results and Discussion}

\begin{figure*}[t]
\includegraphics[width=12cm]{figures/paper_figures/f2.pdf}
\caption{Time series of annual average MDA8O\textsubscript{3}, O\textsubscript{3} and NO\textsubscript{2} for urban sites in Europe and the USA, for the time series types detailed in (table \ref{tab:ts_types})}
\label{fig:conc_plot}
\end{figure*}

As an initial overview, figure \ref{fig:conc_plot} shows the annual mean concentrations across the range of all the urban sites and time series types used in the this study (table \ref{tab:ts_types}). European O\textsubscript{3} concentrations show a steady increase over the last 20 years, with the average across all data increasing from 21.2 ppbv in 2000 to 25.9 ppbv in 2019, the MDA8O\textsubscript{3} increasing from 30.8 to 35.3 ppbv and warm season MDA8O\textsubscript{3}. In the USA average O\textsubscript{3} has increased from 27.7 to 29.7 ppbv, but MDAO\textsubscript{3} and warm season MDA8O\textsubscript{3} both decreased, from 41.5 to 40.9 and 49.9 to 45.1 ppbv respectively. Annual average urban NO\textsubscript{2} is also shown to provide some context to precursor concentrations. Both Europe and the USA have seen reductions in NO\textsubscript{2}, 19.0 in 2000 to 13.0 ppbv in 2019 in Europe and 19.7 to 9.8 ppbv in the USA. The USA did see a minima in 2020 of 9.3 ppbv, but increased to 11.8 ppbv in 2022, whereas Europe continued its decrease to 10.6 ppbv in 2022. 

The other features that do stand out appear in the sub groups containing the warm season, but are most clear in the warm MDA8O\textsubscript{3} series correspond to years with significant heatwaves - 2003, 2006, 2015 and 2018 in Europe and 2012 in the USA. Heatwaves are well documented to increase O\textsubscript{3} concentrations, and are expected to increase in frequency \citep{Schär2004, Russo_2015, https://doi.org/10.1002/2016GL068432, Otero_2016, GOULDSBROUGH2022118975}. The continued impact of heat waves on high O\textsubscript{3} is however, dependant on the future levels of O\textsubscript{3} precursors and reductions and indeed this can be observed at some sites \citep{Meehl_2018, OTERO2021118334, acp-25-2725-2025, acp-25-5101-2025}, but urban sites may take longer to reach levels where O\textsubscript{3} - temperature sensitivity is reduced, owing to a higher initial concentrations \citep{VazquezSantiago2024}.

% New section from here (BSN)


\subsection{Change Point Assignment} \label{sect:new_mda8_piecewise_types}

Regional differences were observed in the types (QR, PQR\_1, PQR\_2) of change point assigned. Figures \ref{}

\textcolor{red}{Whether a given trend analysis is defined as a standard QR (1 piece), two QRs with one change point (PQR 1 - 2 pieces), or 3 QRs with two change points (PQR 2 - 3 pieces) is described in section XXX.}

Figure \ref{fig:test_scenarioType_tau_count} shows the number of sites that fall into each of these categories for the MDA8 trend dataset. Generally, the US dataset is best described using one QR trend line, without the introduction of a change point (for the 50\textsuperscript{th} percentile data, QR = 69 sites, PQR 1 = 48 sites, PQR 2 = 35 sites). In comparison, for the EU data set, introducing change points generally improves the fit of the data, and very few sites are better described using just one QR trend line (for the 50\textsuperscript{th} percentile data, QR = 5 sites, PQR 1 = 121 sites, PQR 2 = 82 sites). A trend line with one, or sometimes two, change points describes the data well in both Northern and Central Europe, as well as Southern Europe. However, in the US, QR trends with no change point are more common in Eastern and Central US, whereas more trends are well described in Western US with the inclusion of one change point. The regionality of MDA8 trends is discussed in more detail in section XXX.
 \clearpage
 
\begin{figure}[h!]
\includegraphics[width=12cm]{test_figures/test_number_of_sites_scenarioType_tau_mda8_all.png}
\caption{Number of sites with no change point (QR, gold), one change point (PQR\_1, red) and two change points (PQR\_2, green) in the MDA8 data, and in the regions of Northern and Central Europe, Southern Europe, Western US and Hawaii, and Eastern and Central US (clockwise from top left).}
\label{fig:test_scenarioType_tau_count}
\end{figure}


\subsection{Trends in MDA8 Across Quantiles} \label{sect:new_mda8_trends}

Trends in MDA8 for each site at a range of different quantiles were calculated using the methodology described in Section XXX. Figure \ref{fig:p_bar_year_mda8_anom_combined} shows the number of sites with increasing and decreasing trends for a selection of quantiles (5\textsuperscript{th}, 50\textsuperscript{th} and 95\textsuperscript{th}), for each year between 2002 and 2019, as well as the degree of certainty of the trends, in the MDA8 O$_3$ data. Generally, there are a large proportion of high certainty positive trends in the 5\textsuperscript{th} and 50\textsuperscript{th} percentile in the EU data set (range of 34 - 72\% of trends across the 20-year period). In contrast there are fewer trends of high certainty in the 95\textsuperscript{th}, the majority of which are decreasing (between 5 - 25\% across the 20-year period). The warm-season data shows a similar pattern, but with fewer high certainty increasing trends in the 5\textsuperscript{th} and 95\textsuperscript{th} percentile. Many more high certainty increasing trends are observed in the cold season only data, particularly in the 5\textsuperscript{th} and 50\textsuperscript{th} percentile data (range of 29 - 73\% across the 20-year period). The proportion of high certainty increasing trends in the cold season appears to be increasing with time.

In the US MDA8 trend data, the majority of 5\textsuperscript{th} percentile trends are high certainty increasing trends (35 - 53\%) and the 95\textsuperscript{th} percentile is dominated by high certainty negative trends (46 - 76\%), with a mixture found in the 50\textsuperscript{th} percentile. In the warm-season only data, there is a larger contribution of decreasing trends across all percentiles. This is particularly true in the 95\textsuperscript{th} percentile data, where the vast majority of trends are high certainty decreasing trends (55 - 84\%). However, there is a notable change year-on-year in the US dataset that is worth some discussion. Although decreasing trends dominate, and the number of high certainty increasing trends are low (1-11\%), there appears to be a step-change. Between 2000 - 2008, between 1-3\% of trends are high certainty increasing trends. However, after 2008, the number of sites showing a high certainty increasing trend increases rapidly from 2 to 16 by 2016, with the majority of these sites located in Western and Central US, and the top three largest magnitude slopes (medium-large), located in the South. In the cold season data, there is very little year-on-year change in the distribution of increasing and decreasing trends in the 5\textsuperscript{th}, 50\textsuperscript{th} and 95\textsuperscript{th} percentile. high certainty.
% Do we want to talk about these sites? Regionality?

\begin{figure}[h!]
\centering
\includegraphics[width=9cm]{test_figures/significance_bars_freeTau_mda8_anom_all.pdf}

\vspace{1em}

\includegraphics[width=9cm]{test_figures/significance_bars_freeTau_mda8_anom_warm.pdf}

\vspace{1em}

\includegraphics[width=9cm]{test_figures/significance_bars_freeTau_mda8_anom_cold.pdf}

\caption{Time series of trends in MDA8 O\textsubscript{3} for (a) annual, (b) warm season, and (c) cold season, for Europe (top) and the USA (bottom).}
\label{fig:p_bar_year_mda8_anom_combined}
\end{figure}

The distribution of slopes and the corresponding uncertainty of each trend is shown in Figure XXX for the 5\textsuperscript{th}, 50\textsuperscript{th} and 95\textsuperscript{th} percentiles and for the years 2002 and 2022 (additional percentiles and years are found in the supplementary).

%This is where I just don't know which plots to add. Take a look at the three o3_map_freeTau figures in the test_figures folder. This is too much for the paper, right?. 

An overview of the trend distributions at the start and end of the 20-year period are presented here. In this summary, we select all trends from 2002 and 2019, as these are the first and last years in which all sites could theoretically have a change point. We use p-values to describe the certainty of each trend, as defined by \cite{chang2023guidancenotebeststatistical}. The magnitude of each trend is prescribed as "small", "medium", large" or "very large", according to the ranges described in Table \ref{tab:magnitude_description_table}.

\begin{table}[h]
\caption{Description of trend magnitude ranges.}
\begin{tabular}{c|c}
Range / ppb yr$^{-1}$                 & Magnitude description \\ \hline
0 \textless slope \textless 0.5       & small                 \\
0.5 \textless{}= slope \textless 1.5  & medium              \\
1.5 \textless{}= slopes \textless 2.5 & large                 \\
slope \textgreater{}= 2.5             & very large           
\end{tabular}
\label{tab:magnitude_description_table}
\end{table}


% Issue here is, you are comparing start and end, but this is very different for EU and US as the EU goes a bit further beyond COVID than the US. Need to think about the best time periods to present here.

% Across the EU data set, 79\% of sites showed an increasing trend in the 5\textsuperscript{th} and 50\textsuperscript{th} percentiles at the start of the 20-year period, with a smaller percentage (30\%) increasing in the 95\textsuperscript{th} percentile. By the end of the vicennial, this has increased to 89\%, 88\%, and 51\% for the 5\textsuperscript{th} and 50\textsuperscript{th} and 95\textsuperscript{th} percentiles respectively. The number of sites showing decreasing trends has approximately halved in this time period for the 5\textsuperscript{th} and 50\textsuperscript{th} percentiles (from \italics{c.a.} 20\% to 10\%), with a smaller reduction observed in the 95\textsuperscript{th} percentile (from 70\% to 49\%). In comparison, the US data set shows 68\% and 58\% of sites increasing at the start of the vicennial in the 5\textsuperscript{th} and 50\textsuperscript{th} percentiles, changing very little by the end of the time period in the 5\textsuperscript{th} percentile (66\%), and reducing to 50\% of sites in the 50\textsuperscript{th} percentile. A comparable percentage of sites are increasing in the US compared to EU at the start of the time period in the 95\textsuperscript{th} percentile (40\%), but a notably smaller percentage are increasing by the end (33\%). Across the three percentiles presented here, the number of increasing trends has increased between the start and end of the time period in the EU, but decreased in the US.

%Across the EU data set at the start of the century (2002), 81\% and 77\% of slopes showed an increasing trend in the 5\textsuperscript{th} and 50\textsuperscript{th} percentiles respectively, compared to 71\% showing a decreasing trend in the 95\% percentile. By 2019, the number of increasing slopes increased across the three percentiles, to 91\% and 85\% for the 5\textsuperscript{th} and 50\textsuperscript{th} percentile, and from 29\% to 39\% in the 95\textsuperscript{th} percentile. Of the increasing trends at the start of the century, the majority (between 63 - 81\%) of positive trends are leading to a small increase in O$_3$, up to 0.5 ppbV yr$^{-1}$. Between 1 - 4\% of positive trends have a large increasing trend in O$_3$ (> 1.5 pbV yr$^{-1}$) across all percentiles, expect for the 95\textsuperscript{th} percentile where a jump up to 12\% of increasing trends are considered to be large. Despite there being more increasing trends in 2019, a larger proportion of these are small increases, with a maximum of 1\% of increasing trends showing a large upward trend, observed in the 95\textsuperscript{th} percentile. In the 95\textsuperscript{th} percentile, the magnitude of decreasing trends increases from 2002 to 2019, the the percentage of small decreasing trends increasing from 78\% to 95\% of total decreasing trends in that percentile. 

Across the EU data set at the start of the century (2002), 81\% and 77\% of slopes showed an increasing trend in the 5\textsuperscript{th} and 50\textsuperscript{th} percentiles respectively (Both with 46\% at medium certainty or higher), compared to 71\% showing a decreasing trend in the 95\% percentile (24\% at medium certainty or higher). By 2019, the number of increasing slopes increased across the three percentiles, to 91\% and 85\% for the 5\textsuperscript{th} and 50\textsuperscript{th} percentile (72\% and 66\% at medium certainty or higher), and from 29\% to 39\% in the 95\textsuperscript{th} percentile (from 6\% to 13\% at medium certainty or higher). This supports an increase in the number of EU sites with increasing O$_3$ trends at the end of the 20-year period compared to the beginning, for which the majority of sites showed increasing trends in the low-mid percentiles during both time periods. In the 95\textsuperscript{th} percentile, the majority of sites were are showing decreasing trends in both time periods, but a larger proportion of increasing trends is apparent in 2019 compared to 2002. 

In comparison, 75\%  and 55\% of slopes showed an increasing trend in the 5\textsuperscript{th} and 50\textsuperscript{th} percentiles of the US data set at the start of the century (46\% and 36\% at medium certainty or higher). Similarly to the EU, 95\textsuperscript{th} percentile trends were decreasing the in US (74\% - 53\% at medium certainty or higher). However, in contrast to EU, by 2019 fewer trends were increasing, down to 67\% in the 5\textsuperscript{th} percentile (38\% at medium certainty or higher). In the 50\textsuperscript{th} percentile, more trends are decreasing (52\% - 26\% at medium certainty) than increasing (48\% - 33\% at medium certainty or higher), although more trends are increasing than decreasing when the data is filtered for medium certainty or higher. The reduction in the number of trends showing an increasing in O$_3$ in the latter part of the vicennial indicates that generally air quality with respect to O$_3$ in the US is improving. However, marginally more of the medium-high certainty trends are increasing than decreasing in the 50\textsuperscript{th} percentile. In addition to this, there are fewer decreasing trends in the 95\textsuperscript{th} percentile (69\% decreasing - 52\% at moderate certainty or higher). However, the change in the number of moderate-high certainty decreasing trends is minimal.

Separating the MDA8 trends into warm (April - September) and cold (October - March) seasons reveals much clearer trends, and reveals a regionality in the direction of MDA8 trends (section XXX). Generally across EU, there more increasing trends in the cold season (72\% total;  32\% of moderate or higher certainty) compared to the warm season (27\%, total; 6\% of moderate or higher certainty) in the 95\textsuperscript{th} percentile. In contrast, although there are a similar number of increasing trends in the warm and cold season for the 5\textsuperscript{th} percentile (81\% and 83\% respectively), there are many more increasing trends with a moderate or higher certainty and medium or greater magnitude in the warm season (20\%) compared to the cold season (3\%). This pattern is retained in the 2019 slope data, with 18\% of 5\textsuperscript{th} percentile trends increasing at moderate or higher certainty and medium or higher magnitude in the warm season, and an increase in the number of increasing trends across all three percentiles in the 2019 data for the cold season. In summary, although we are seeing increasing trends across the percentiles in both 2002 and 2019, this is particularly prevalent for the lower values in the warmer months, where the trends are not only increasing, but are of higher certainty and magnitude.

In the US cold season at the start of the century, 85\%, 80\% and 50\% of trends are increasing in the 5\textsuperscript{th}, 50\textsuperscript{th}, and 95\textsuperscript{th} percentiles respectively, the majority of which are of moderate or high certainty in the 5\textsuperscript{th} and 50\textsuperscript{th} percentile. In comparison, there are fewer increasing trends in the warm season for the 5\textsuperscript{th} (51\%), 50\textsuperscript{th} (27\%) and 95\textsuperscript{th} (14\%) percentiles. In the warm season, the majority of trends are decreasing in the 95\textsuperscript{th} percentile (86\%), with 49\% of all trends in this percentile having a moderate or higher degree of certainty, at a medium or larger magnitude. The majority of 50\textsuperscript{th} percentile trends are also decreasing (73\%; 60\% at moderate or higher certainty), but with a smaller proportion with both moderate or higher certainty and medium or higher magnitude (12\%). Comparable trends are observed in the 2019 slopes, but with a small increase in the number of increasing trends of moderate certainty or higher in the warm season 50\textsuperscript{th} and 95\textsuperscript{th} percentile, and in the 95\textsuperscript{th} percentile cold season. In the cold season, the number of increasing trends with a moderate or high certainty and also medium or higher magnitudes in the 95\textsuperscript{th} percentile has increased from 6\% to 10\% of the trends.

\subsection{Cluster stuff} \label{sect:cluster_stuff}
Hierarchical clustering (HC) (described in section XXX) was used to identify timeseries with similar features. Here, we discuss the clustering observed in the MDA8 O$_3$ data at the 5\textsuperscript{th}, 50\textsuperscript{th} and 95\textsuperscript{th} percentile. HC was performed for each of the three types of MDA8 O$_3$ data set (all, warm season only, cold season only), separately for the United States of America and Europe. 

The HC of the European data set indicates that most sites are well described by one cluster only, in the 5\textsuperscript{th} and 50\textsuperscript{th} percentiles across all three types of MDA8 data aggregation. However, a few additional clusters are introduced in the 5\textsuperscript{th} and 50\textsuperscript{th} percentile cold-season data, as well as the 50\textsuperscript{th} percentile full data set and warm-season only data set. In the cold-season, these clusters don't appear to have any particular regionality in the 5\textsuperscript{th} percentile. However, in the 50\textsuperscript{th} percentile two new clusters are introduced, localised around South Central and East Europe (clusters 3 and 4). In the 95\textsuperscript{th} percentile (high extreme) data, more regional clustering is observed in the full MDA8 data, and the warm-season only data. This regional clustering is more clearly observed in the MDA8 data, with clusters location in North Eastern Europe (cluster 1); across the continent from South and Central Western France to East Europe (cluster 2); some selected sites from across this same area, but also including sites in Spain and the UK (cluster 3); Northern France (cluster 4), the Far East (cluster 5), and North East (cluster 6). In the warm season, the majority of clustered sites are located across Europe, with some exceptions in the North East (cluster 2), North West (cluster 3), and Spain (cluster 4).

The HC of the United States of America data set has only one large cluster for the full MDA8 data set, cold-season, and warm-season only data in the 5\textsuperscript{th} percentile. However, some clear regional clusters appear in the 50\textsuperscript{th} and 95\textsuperscript{th} percentile data sets across all three data types. The main cluster sites are located in Florida, also including some sites along the Gulf of Mexico and the South in the cold-season (cluster 2, or cluster 3 in the MDA8 full 95\textsuperscript{th} percentile data); around Los Angeles (clusters 5 (in the 5\textsuperscript{th} percentile percentile; 6 (in the full MDA8 data set, 95\textsuperscript{th} percentile), 7 (in the full MDA8 dataset); 8 and 10. Some clustering around the Rocky Mountains and the South West is observed in the full MDA8 data set in the high extremes (cluster 2, 95\textsuperscript{th} percentile). The East Coast generally falls into the largest cluster which doesn't appear to have a strong regonality, except for the in full MDA8 data 95\textsuperscript{th} percentile, where there is an independent North East coast cluster (cluster 5). Also in this data type, and in the 95\textsuperscript{th} percentile data, there is a cluster located in South Central USA (cluster 1).




% 

% New section ends here (BSN)

\subsection{Overview of Trends} \label{sect:overview_of_trends}
\begin{figure*}[p]
\includegraphics[width=12cm]{figures/f3_eu_arrows.pdf}
\caption{Summary of trends across Europe. Angle of arrow indicates magnitude and direction, with the following orientations: vertical up - +2.5 ppb yr\textsuperscript{-1}, horizontal - 0 ppb yr\textsuperscript{-1} and vertically down - -2.5 ppb yr\textsuperscript{-1}. The few sites with absolute trends > 2.5 ppb yr\textsuperscript{-1} have been clamped to $\pm$ 2.5 ppb yr\textsuperscript{-1}. Colour indicates significance and direction, darkest colours being most significant (p < 0.05) and green the least (p > 0.33). Blues are decreasing trends and reds are increasing trends. The upper 4 plots show O\textsubscript{3} trends, and the lower 4 show NO\textsubscript{2} trends. Each group of 4 groups the trends into segments of the study period. If a change point occurs in a group an arrow for both trends is shown.}

\label{fig:arrow_eu}
\end{figure*}

\begin{figure*}[p]
\includegraphics[width=12cm]{figures/f4_usa_arrows.pdf}
\caption{Summary of trends across the United States of America. Angle of arrow indicates magnitude and direction, with the following orientations: vertical up - +2.5 ppb yr\textsuperscript{-1}, horizontal - 0 ppb yr\textsuperscript{-1} and vertically down - -2.5 ppb yr\textsuperscript{-1}. The few sites with absolute trends > 2.5 ppb yr\textsuperscript{-1} have been clamped to $\pm$ 2.5 ppb yr\textsuperscript{-1}. Colour indicates significance and direction, darkest colours being most significant (p < 0.05) and green the least (p > 0.33). Blues are decreasing trends and reds are increasing trends. The upper 4 plots show O\textsubscript{3} trends, and the lower 4 show NO\textsubscript{2} trends. Each group of 4 groups the trends into segments of the study period. If a change point occurs in a group an arrow for both trends is shown.}
\label{fig:arrow_us}
\end{figure*}

The $\tau$ = 0.5 trends of O\textsubscript{3} and NO\textsubscript{2} were examined and are presented on maps in figures \ref{fig:arrow_eu} and \ref{fig:arrow_us}.

In Europe, between 2000 and 2004 there were 108 trends where O\textsubscript{3} was increasing, and 41 where it was decreasing, 32 showed no trend. By 2015-2021, 123 had an increasing trend, 32 decreasing and 41 with no trend, indicating generally increasing O\textsubscript{3} concentrations across Europe. This is somewhat mirrored by the trends in NO\textsubscript{2}, where in 2000-2004 69 trends where increasing, 181 decreasing and 59 with no trend, changing to 29 increasing, 241 decreasing and 39 with no trend by 2015-2021. This would suggest a general picture that urban locations in Europe are still on the VOC limited portion of the O\textsubscript{3} production isopleth, and decreasing NO\textsubscript{x} increases O\textsubscript{3} production, however, it is not quite as ubiquitous as this, as shown in \ref{sect:2020_in_europe} which discusses how O\textsubscript{3} responded to changing NO\textsubscript{2} following restrictions in 2020. 

This contrasts with the USA, were in 2000-2004, there were 10 sites with increasing NO\textsubscript{2}, and 0 sites increasing between 2005 and 2014, but in 2015-2021 17 sites had begun increasing in NO\textsubscript{2} again – though notably as seen from figure \ref{fig:arrow_us} these are different sites to those that were increasing at the beginning of the century. This increase in NO\textsubscript{2} is not clearly reflected by the O\textsubscript{3} trends with 62 increasing, 23 decreasing and 22 showing no trend in 2000 – 2004 and 44 increasing, 34 decreasing, and 25 showing no trend in 2015-2021. A decreasing number of sites with a positive trend in O\textsubscript{3}, suggests that the majority of decreasing NO\textsubscript{2} trends are slowing the increase in O\textsubscript{3}. The counts of sites at all values of $\tau$ can be found in tables \ref{table:europe_slope_segs} and \ref{table:usa_slope_segs}. 

Across the entire study period, for O\textsubscript{3} in Europe, at sites with increasing trends the median slope was 0.36 ppb yr\textsuperscript{-1}, slightly lower than in the US, where they were increasing by 0.40 ppb yr\textsuperscript{-1}. For NO\textsubscript{2}, the median rate of increase was the same in Europe (0.36 ppb yr\textsuperscript{-1}), but higher at 0.71 ppb yr\textsuperscript{-1} in the US. For decreasing trends, O\textsubscript{3} and NO\textsubscript{2} in Europe and the US were all similar, O\textsubscript{3} in Europe decreasing at -0.40 ppb yr\textsuperscript{-1}, (-0.36 ppb yr\textsuperscript{-1}) in the US and for NO\textsubscript{2} -0.39 ppb yr\textsuperscript{-1} in Europe and -0.47 in the US. This information along with the 5 th and 95 th percentile (still at $\tau$ = 0.5) are shown in table \ref{table:slope_ranges}. 

\input{tables/t1_slope_ranges.txt}

\input{tables/t2_europe_segs_11_14.txt}

\input{tables/t3_usa_segs_11_14.txt}

\clearpage
\subsubsection{Significance of Trends} \label{sect:significance_of_trends}

\begin{figure*}[htbp]
\includegraphics[width=12cm]{figures/f5_significance_bars.pdf}
\caption{Time series of trends in O\textsubscript{3} (left) and NO\textsubscript{2} (right) for Europe (top) and the USA (bottom). Positive trends are contained in the bar above 0 and negative trends are contained in the bar below 0. Colour indicates significance and direction, darkest colours being most significant (p < 0.05) and green the least (p > 0.33). Blues are decreasing trends and reds are increasing trends. In this case trends with p > 0.33 have been sorted by their direction.}
\label{fig:p_bar_year}
\end{figure*}

Figure \ref{fig:p_bar_year} counts the trends by significance and direction per year. 2000, 2001 and 2021 onward in Europe and 2000, 2001, 2020 and 2021 onwards for the USA were excluded from this year on year analysis, as it is not possible to capture changing trends in these regions due to the limitations placed on where change points can occur as detailed in section \ref{sect:method}.

This reveals the majority of trends are of high certainty. It also reinforces the observations from section \ref{sect:overview_of_trends}, where both Europe and the US see O\textsubscript{3} increasing at number of sites, with there being proportionally more sites with a decreasing trend in the US. This additionally shows that the number of high certainty decreasing trends in O\textsubscript{3} has been present since \textasciitilde{2010}, joined by a slow decline in increasing trends. 

For NO\textsubscript{2} the feature of some sites beginning to show increasing trends in the US is clear, with this pattern beginning in 2015-2016. Between 2000 and 2019 nearly all European sites moved to a trend of decreasing NO\textsubscript{2}, however, this reversed for several sites in 2020. This trend reversal can be attributed to the reduction in NO\textsubscript{2} concentrations in Europe which were a side effect of public health measures \citep{acp-20-15743-2020, acp-21-4169-2021}, and is discussed in more detail in \ref{sect:2020_in_europe}.  

\section{Conclusions}  %% \conclusions[modified heading if necessary]
%Trends in urban O\textsubscript{3} and NO\textsubscript{2} across the United States of America and Europe were explored between 2000 - 2021. Before exploring the trends, a broad look at the median change in mixing ratios across all sites in the USA and Europe revealed an increase in median O\textsubscript{3} mixing ratios in both Europe (+8 ppb) and the USA (+4 ppb) between 2000 and 2019. NO\textsubscript{2} median mixing ratios generally decreased over the 2000 - 2021 period, but a noticeable up-tick in NO\textsubscript{2} was observed in the USA in later years. 

Trends in urban O\textsubscript{3} and NO\textsubscript{2} were calculated from de-seasoned monitoring site data across both Europe and the USA. A quantile regression analysis was utilised to assess long-term trends, with the method extended to a piecewise approach. This allowed for some freedom to capture changes in a complex dataset, whilst being able to describe the trends with a small number of coefficients. Break points were limited to two per time series to keep the focus on long-term trends and the largest changes. One limitation to this approach is that time series with 3 or more large changes will not be well described by this method - however, after inspection of the regressions and comparison with a LOESS, two points were found to capture the structure well. The trends constructed from the piecewise regressions were evaluated by their statistical significance, and those poorly described using this methodology ranked lower in significance and were subsequently investigated less in the analysis. Piecewise segments were required to have $\ge$ 2 years of data, which limited the ability to define the most recent changes in a given time series, and such is why only Europe was analysed for the effects of the COVID-19 restrictions. 

In Europe, more sites were found to have an increasing O\textsubscript{3} trend between 2015-2021, compared to 2000-2004, with NO\textsubscript{2} trends showing the reverse effect (fewer sites increasing, and more decreasing in 2015-2021 compared to 2000- 2004). This broadly suggests that a VOC-limited relationship for O\textsubscript{3} formation could be common across urban sites in Europe. In the USA, the reverse is true, with a reduction in the number of sites with increasing O\textsubscript{3} trends in 2015-2021 compared with the 2000-2004 period, but with more sites with increasing NO\textsubscript{2} in 2015-2021 compared with 2000-2004. The majority of O\textsubscript{3} trends are positive and of high certainty (p <= 0.05) in Europe across the 20-year period. This is also broadly true for the USA, but the proportion of sites showing this trend reduces with time, alongside an increase in the number of sites with high-certainty negative trends in O\textsubscript{3} between 2010-2021. Previous studies have attributed positive urban O\textsubscript{3} trends in the USA to increasing winter O\textsubscript{3} concentrations \citep{Simon_2015}. However, differences in seasonality would not be observed in this analysis as the trends are derived from de-seasoned data. From the trends described here, the median ($\pm$ median absolute deviation) change in O\textsubscript{3} over the last two decades were calculated as 5.0 $\pm$ 4.6 ppb in Europe and 2.4 $\pm$ 4.9 ppb in the USA, with corresponding changes in NO\textsubscript{2} as -6.8 $\pm$ 3.6 and -8.4 ppb. 

Given the nature of its long lifetime and the complexity of its chemistry, when describing trends in O\textsubscript{3} it is important to look beyond the 50th percentile. A study of European ozone trends between 1995-2014 found that whilst the 50th percentile data did not show significant trends, significant trends were observed in the 5 th and 95 th percentiles ($\tau$ values) \citep{acp-18-5589-2018}. An exploration of the $\tau$ values in this study revealed smaller reductions in trend-derived change in O\textsubscript{3} mixing ratios in 2021 since 2000 for the lower $\tau$ values compared to higher values in Europe. This, coupled with the fact that generally O\textsubscript{3} trends across Europe are positive, suggests that the lowest ambient O\textsubscript{3} mixing ratios are increasing more rapidly than the highest values. This is consistent with the findings of \cite{acp-18-5589-2018}, who found that although peak O\textsubscript{3} concentrations were coming down across urban and suburban sites between 1995 - 2014 in Europe, an increase in the lower O\textsubscript{3} concentrations was observed in more than 75 \% of sites. 

An assessment of the spatio-temporal distribution of the first and second change points revealed more sites with a directional switch in O\textsubscript{3} trend across Europe were in the positive to negative direction in the first change point, but with more sites with a negative to positive second change point. In the USA dataset, the biggest switches from negative to positive in the first change point occurred in California and Florida, but a reversal of this trend from positive to negative was also observed across these states in the second change point. To supplement this, a calculation of the 4th highest daily maximum 8-hour running mean for O\textsubscript{3} (4MDA8) revealed high 4MDA8 hotspots in parts of southern Europe, but particularly in California, consistent with previous findings of \cite{fleming_2018}. Furthermore, the USA has seen significant increasing trends in NO\textsubscript{2} since 2015 at several sites. However, there are limited sites that have been used in this study with co-located NO\textsubscript{2} and O\textsubscript{3}, and the O\textsubscript{3} response is not consistent. This is not unexpected due to the wide range of conditions that an ‘urban’ site can be found in (e.g. variations in local VOC concentrations), but it should be expected that in some locations O\textsubscript{3} will be sensitive to this increase in NO\textsubscript{2}, especially if it persists.

The impact of COVID-19 on ambient mixing ratios of NO\textsubscript{2} is well studied, with one previous study estimating that NO\textsubscript{2} concentrations across urban background sites in Europe were 32 \% lower than expected in 2020 \citep{acp-21-4169-2021}. The impact of this COVID-19 effect was observed in the trend analysis of Europe described in this study, with the vast majority of directionally switching second change points in NO\textsubscript{2} occurring in 2020 in the negative to positive direction. This can be attributed to lower ambient NO\textsubscript{2} mixing ratios in 2020, leading to an increasing trend in the subsequent years as business-as-usual conditions resumed.  These increasing NO\textsubscript{2} concentrations have in some cases been accompanied by increasing O\textsubscript{3}/O\textsubscript{x} and should continue to be studied, since in many cases the prior downward trend in NO\textsubscript{2} has not resumed.


\clearpage

%% The following commands are for the statements about the availability of data sets and/or software code corresponding to the manuscript.
%% It is strongly recommended to make use of these sections in case data sets and/or software code have been part of your research the article is based on.

%\codeavailability{TEXT} %% use this section when having only software code available


%\dataavailability{TEXT} %% use this section when having only data sets available


\codedataavailability{All data for this study can be downloaded from the relevant databases. 
Code for performing the download as well as the analysis can be found at 10.5281/zenodo.14538198 \citep{drysdale_2024_14538198}} %% use this section when having data sets and software code available


%\sampleavailability{TEXT} %% use this section when having geoscientific samples available


%\videosupplement{TEXT} %% use this section when having video supplements available


%\appendix
%\section{}    %% Appendix A

%\noappendix       %% use this to mark the end of the appendix section. Otherwise the figures might be numbered incorrectly (e.g. 10 instead of 1).

%% Regarding figures and tables in appendices, the following two options are possible depending on your general handling of figures and tables in the manuscript environment:

%% Option 1: If you sorted all figures and tables into the sections of the text, please also sort the appendix figures and appendix tables into the respective appendix sections.
%% They will be correctly named automatically.

%% Option 2: If you put all figures after the reference list, please insert appendix tables and figures after the normal tables and figures.
%% To rename them correctly to A1, A2, etc., please add the following commands in front of them:

\appendixfigures  %% needs to be added in front of appendix figures

\appendixtables   %% needs to be added in front of appendix tables

%% Please add \clearpage between each table and/or figure. Further guidelines on figures and tables can be found below.



\authorcontribution{BSN and WSD equally contributed to all aspects of this manuscript's production} %% this section is mandatory

\competinginterests{The authors declare that they have no conflict of interest.} %% this section is mandatory even if you declare that no competing interests are present

\begin{acknowledgements}
The Viking cluster was used during this project, which is a high performance compute facility provided by the University of York. We are grateful for computational support from the University of York, IT Services and the Research IT team.

The Authors acknowledge Prof. James Lee for their scientific advice and Prof. David Carslaw and Dr Stuart Lacy for their advice on the statistical analysis. We also thank Dr Stuart Lacy for their help with SQL and very useful suggestions on managing large datasets. 
\end{acknowledgements}


%% REFERENCES

%% The reference list is compiled as follows:



%% Since the Copernicus LaTeX package includes the BibTeX style file copernicus.bst,
%% authors experienced with BibTeX only have to include the following two lines:
%%
 \bibliographystyle{copernicus}
 \bibliography{references}
%%
%% URLs and DOIs can be entered in your BibTeX file as:
%%
%% URL = {http://www.xyz.org/~jones/idx_g.htm}
%% DOI = {10.5194/xyz}


%% LITERATURE CITATIONS
%%
%% command                        & example result
%% \citet{jones90}|               & Jones et al. (1990)
%% \citep{jones90}|               & (Jones et al., 1990)
%% \citep{jones90,jones93}|       & (Jones et al., 1990, 1993)
%% \citep[p.~32]{jones90}|        & (Jones et al., 1990, p.~32)
%% \citep[e.g.,][]{jones90}|      & (e.g., Jones et al., 1990)
%% \citep[e.g.,][p.~32]{jones90}| & (e.g., Jones et al., 1990, p.~32)
%% \citeauthor{jones90}|          & Jones et al.
%% \citeyear{jones90}|            & 1990



%% FIGURES

%% When figures and tables are placed at the end of the MS (article in one-column style), please add \clearpage
%% between bibliography and first table and/or figure as well as between each table and/or figure.

% The figure files should be labelled correctly with Arabic numerals (e.g. fig01.jpg, fig02.png).


%% ONE-COLUMN FIGURES

%%f
%\begin{figure}[t]
%\includegraphics[width=8.3cm]{FILE NAME}
%\caption{TEXT}
%\end{figure}
%
%%% TWO-COLUMN FIGURES
%
%%f
%\begin{figure*}[t]
%\includegraphics[width=12cm]{FILE NAME}
%\caption{TEXT}
%\end{figure*}
%
%
%%% TABLES
%%%
%%% The different columns must be seperated with a & command and should
%%% end with \\ to identify the column brake.
%
%%% ONE-COLUMN TABLE
%
%%t
%\begin{table}[t]
%\caption{TEXT}
%\begin{tabular}{column = lcr}
%\tophline
%
%\middlehline
%
%\bottomhline
%\end{tabular}
%\belowtable{} % Table Footnotes
%\end{table}
%
%%% TWO-COLUMN TABLE
%
%%t
%\begin{table*}[t]
%\caption{TEXT}
%\begin{tabular}{column = lcr}
%\tophline
%
%\middlehline
%
%\bottomhline
%\end{tabular}
%\belowtable{} % Table Footnotes
%\end{table*}
%
%%% LANDSCAPE TABLE
%
%%t
%\begin{sidewaystable*}[t]
%\caption{TEXT}
%\begin{tabular}{column = lcr}
%\tophline
%
%\middlehline
%
%\bottomhline
%\end{tabular}
%\belowtable{} % Table Footnotes
%\end{sidewaystable*}
%
%
%%% MATHEMATICAL EXPRESSIONS
%
%%% All papers typeset by Copernicus Publications follow the math typesetting regulations
%%% given by the IUPAC Green Book (IUPAC: Quantities, Units and Symbols in Physical Chemistry,
%%% 2nd Edn., Blackwell Science, available at: http://old.iupac.org/publications/books/gbook/green_book_2ed.pdf, 1993).
%%%
%%% Physical quantities/variables are typeset in italic font (t for time, T for Temperature)
%%% Indices which are not defined are typeset in italic font (x, y, z, a, b, c)
%%% Items/objects which are defined are typeset in roman font (Car A, Car B)
%%% Descriptions/specifications which are defined by itself are typeset in roman font (abs, rel, ref, tot, net, ice)
%%% Abbreviations from 2 letters are typeset in roman font (RH, LAI)
%%% Vectors are identified in bold italic font using \vec{x}
%%% Matrices are identified in bold roman font
%%% Multiplication signs are typeset using the LaTeX commands \times (for vector products, grids, and exponential notations) or \cdot
%%% The character * should not be applied as mutliplication sign
%
%
%%% EQUATIONS
%
%%% Single-row equation
%
%\begin{equation}
%
%\end{equation}
%
%%% Multiline equation
%
%\begin{align}
%& 3 + 5 = 8\\
%& 3 + 5 = 8\\
%& 3 + 5 = 8
%\end{align}
%
%
%%% MATRICES
%
%\begin{matrix}
%x & y & z\\
%x & y & z\\
%x & y & z\\
%\end{matrix}
%
%
%%% ALGORITHM
%
%\begin{algorithm}
%\caption{...}
%\label{a1}
%\begin{algorithmic}
%...
%\end{algorithmic}
%\end{algorithm}
%
%
%%% CHEMICAL FORMULAS AND REACTIONS
%
%%% For formulas embedded in the text, please use \chem{}
%
%%% The reaction environment creates labels including the letter R, i.e. (R1), (R2), etc.
%
%\begin{reaction}
%%% \rightarrow should be used for normal (one-way) chemical reactions
%%% \rightleftharpoons should be used for equilibria
%%% \leftrightarrow should be used for resonance structures
%\end{reaction}
%
%
%%% PHYSICAL UNITS
%%%
%%% Please use \unit{} and apply the exponential notation


\end{document}
