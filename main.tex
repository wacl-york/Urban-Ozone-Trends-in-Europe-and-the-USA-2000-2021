%% Copernicus Publications Manuscript Preparation Template for LaTeX Submissions
%% ---------------------------------
%% This template should be used for copernicus.cls
%% The class file and some style files are bundled in the Copernicus Latex Package, which can be downloaded from the different journal webpages.
%% For further assistance please contact Copernicus Publications at: production@copernicus.org
%% https://publications.copernicus.org/for_authors/manuscript_preparation.html


%% Please use the following documentclass and journal abbreviations for preprints and final revised papers.

%% 2-column papers and preprints
\documentclass[journal abbreviation, manuscript]{copernicus}



%% Journal abbreviations (please use the same for preprints and final revised papers)


% Advances in Geosciences (adgeo)
% Advances in Radio Science (ars)
% Advances in Science and Research (asr)
% Advances in Statistical Climatology, Meteorology and Oceanography (ascmo)
% Aerosol Research (ar)
% Annales Geophysicae (angeo)
% Archives Animal Breeding (aab)
% Atmospheric Chemistry and Physics (acp)
% Atmospheric Measurement Techniques (amt)
% Biogeosciences (bg)
% Climate of the Past (cp)
% DEUQUA Special Publications (deuquasp)
% Earth Surface Dynamics (esurf)
% Earth System Dynamics (esd)
% Earth System Science Data (essd)
% E&G Quaternary Science Journal (egqsj)
% EGUsphere (egusphere) | This is only for EGUsphere preprints submitted without relation to an EGU journal.
% European Journal of Mineralogy (ejm)
% Fossil Record (fr)
% Geochronology (gchron)
% Geographica Helvetica (gh)
% Geoscience Communication (gc)
% Geoscientific Instrumentation, Methods and Data Systems (gi)
% Geoscientific Model Development (gmd)
% History of Geo- and Space Sciences (hgss)
% Hydrology and Earth System Sciences (hess)
% Journal of Bone and Joint Infection (jbji)
% Journal of Micropalaeontology (jm)
% Journal of Sensors and Sensor Systems (jsss)
% Magnetic Resonance (mr)
% Mechanical Sciences (ms)
% Natural Hazards and Earth System Sciences (nhess)
% Nonlinear Processes in Geophysics (npg)
% Ocean Science (os)
% Polarforschung - Journal of the German Society for Polar Research (polf)
% Primate Biology (pb)
% Proceedings of the International Association of Hydrological Sciences (piahs)
% Safety of Nuclear Waste Disposal (sand)
% Scientific Drilling (sd)
% SOIL (soil)
% Solid Earth (se)
% State of the Planet (sp)
% The Cryosphere (tc)
% Weather and Climate Dynamics (wcd)
% Web Ecology (we)
% Wind Energy Science (wes)


%% \usepackage commands included in the copernicus.cls:
%\usepackage[german, english]{babel}
%\usepackage{tabularx}
%\usepackage{cancel}
%\usepackage{multirow}
%\usepackage{supertabular}
%\usepackage{algorithmic}
%\usepackage{algorithm}
%\usepackage{amsthm}
%\usepackage{float}
%\usepackage{subfig}
%\usepackage{rotating}

\usepackage{booktabs} 
\usepackage{adjustbox}
\usepackage{longtable}

\begin{document}

\title{Urban Ozone Trends in Europe and the USA (2000-2021)}


% \Author[affil]{given_name}{surname}

\Author[1,2,*][beth.nelson@york.ac.uk]{Beth S.}{Nelson} %% correspondence author
\Author[1,2,*][will.drysdale@york.ac.uk]{Will S.}{Drysdale}

\affil[1]{Wolfson Atmospheric Chemistry Laboratories, Department of Chemistry, University of York, Heslington, York, YO10 5DD, UK}
\affil[2]{National Centre for Atmospheric Science, University of York, York, UK}
\affil[*]{These authors contributed equally to this work.}
%% The [] brackets identify the author with the corresponding affiliation. 1, 2, 3, etc. should be inserted.

%% If an author is deceased, please mark the respective author name(s) with a dagger, e.g. "\Author[2,$\dag$]{Anton}{Smith}", and add a further "\affil[$\dag$]{deceased, 1 July 2019}".

%% If authors contributed equally, please mark the respective author names with an asterisk, e.g. "\Author[2,*]{Anton}{Smith}" and "\Author[3,*]{Bradley}{Miller}" and add a further affiliation: "\affil[*]{These authors contributed equally to this work.}".


\runningtitle{TEXT}

\runningauthor{TEXT}





\received{}
\pubdiscuss{} %% only important for two-stage journals
\revised{}
\accepted{}
\published{}

%% These dates will be inserted by Copernicus Publications during the typesetting process.


\firstpage{1}

\maketitle



\begin{abstract}
TEXT
\end{abstract}


\copyrightstatement{TEXT} %% This section is optional and can be used for copyright transfers.


\section{Introduction}  %% \introduction[modified heading if necessary]
Tropospheric ozone is a greenhouse gas and air pollutant harmful to human health, and plant growth (Fleming et al., 2018; Mills et al., 2018; Szopa et al., 2021). It is a secondary air pollutant, formed from the photochemical reactions of primary pollutants NOx (NO + NO2) and volatile organic compounds (VOCs). Despite global successes in reducing primary pollutant emissions over the past few decades, global exposure to ozone has been increasing throughout the 21st century. This is particularly observed in urban areas, where the vast majority of the global population live, projected to increase to 68\% in 2050 from 55\% in 2018 (UN, 2019). In a study of 12946 cities located worldwide, the average mean weighted ozone concentration increased by 11\% between 2000 and 2019, and the number of cities exceeding the WHO peak season ozone standard increased from 89\% to 96\% (Malashock et al., 2022).

Due to the complexity of ozone production, its trend direction, magnitude, and significance varies by location. A previous study calculated trends in two ozone metrics globally over the period of 2000 - 2014: 4th highest daily maximum 8-hour ozone (4MDA8), and the number of days with MDA8 > 70 ppb ozone (NDGT70) (Fleming et al., 2018). The study used data from 4801 global monitoring sites over this time period. For both of these metrics, downward trends were observed for most of the USA, and some sites in Europe. However, over the period of 2010-2014 (2,600 sites utilised), sites located in regions with the highest ozone precursor emissions across North America, Europe and East Asia had the highest values in 4MDA8 and NDGT70. In North America and Europe, this was particularly true for California and parts of southern Europe (Fleming et al., 2018).

Since the 1990s, a general downward trend in urban ozone pollution has been observed in the United States (He et al., 2020). This reducing trend has been linked to stricter limiting regulations on the emissions of primary pollutants such as NOx and VOCs. Although NOx and VOC emissions in Europe have also been declining since the late 1980s, the trend in ozone is less clear due to large inner-annual variation, driven by climate variability and the dispersion and transport of pollutants from other regions (Jonson et al., 2006; Yan et al., 2018). Between 1995 and 2014, negative trends in the highest ozone levels across urban sites in Europe were identified due to pollutant emission restrictions across Europe. However, increasing background levels, particularly in northern and eastern Europe, make it difficult to identify strong trends in urban ozone when transboundary effects are considered. (Yan et al., 2018). Despite this, a study of 93 suburban and urban sites across Europe identified notable enhancements in ozone seasonal and annual means between 1995 - 2012, even with the continuous downward trend in anthropogenic emissions across the continent (Yan et al., 2018).

Since these earlier studies, much of the literature focuses on the impact of the COVID-19 pandemic on air pollutant concentrations and trends (Lee et al., 2020; Shi et al., 2021; Grange et al., 2021). A study of eleven cities across the world, all of which implemented stringent lockdown measures in early 2020, captured sudden decreases in deweathered NO2 concentrations, concurrent with sudden increases in O3, in most locations. (Shi et al., 2021). Another study employing machine learning models to predict business-as-usual levels of pollutants, estimated that NO2 concentrations were 32\% lower than expected across 102 European urban background locations, whereas O3 was 21\% higher (Grange et al., 2021). It is highly likely that this global event has resulted in perturbations of both long-term NO2 and O3 trends, particularly in locations where lockdowns were stringent or lengthy.

This study examines the trends in urban O3 and NO2 using monitoring site data from 2020 - 2023. It employs quantile regression and change point detection to construct trends that capture the broad structure of a complex time series, while remaining explainable with consists statistics. Section \ref{sect:method} details the method used to create the trends, section \ref{sect:overview_of_trends} provides a high-level overview of the trends with section \ref{sect:significance_of_trends} focusing on the significance of the trends and highlighting features that are discussed in detail later on. Section \ref{sect:trends_across_quantiles} explores the information that can be gained from trends calculated over a range of quantiles and finally section \ref{sect:case_studies} provides details on some of the features revealed earlier in the analysis. 

% To understand blah

\section{Methodology} \label{sect:method}
Hourly NO2 and O3 data were obtained for urban sites in the USA and Europe using the TOAR-II (REF) and European Environment Agency (EEA) (REF) databases, using the r-packages toarR (REF) and saqgetr (REF) respectively. TOAR-II data was retrieved between 2000-01-01 and 2021-12-31 (the latest available) and EEA data between 2000-01-01 and 2023-12-31, the latter being extended longer so the effects of the COVID-19 pandemic could be observed more clearly. Time series were required to have 90 \% data coverage between 2000-01-01 and 2021-12-31 and retained time series were averaged to daily medians. Additionally, a small number of time series were removed following visual inspection - these are listed in table SI.XX. This resulted in 228 O3 time series (144 Europe, 84 USA), 322 NO2 time series (245 Europe, 77 USA) and 126 Ox (NO2 + O3) time series (101 Europe, 25 USA).

Firstly, the time series were de-seasoned by calculating a monthly median climatology and subtracting this from the time series, producing an ‘anomaly’ time series. The NO2 and O3 anomaly time series were summed to produce Ox anomaly time series for sites that have both NO2 and O3 data available. Following this, several methods were defined to help describe trends. As the time series are non-linear Locally Estimated Scatterplot Smoothing (LOESS) was applied first. This method works by calculating many least squared fits over a moving weighted subset of the data – with this subset tuned based on the desired ‘smoothness’ of the result. In this case with a smoothing parameter ($\alpha$) chosen as 0.5, which was determined by inspection to capture a balance of broad trends and medium-term non-linearity. LOESS works well for drawing the eye to features of the time series but does not allow for broad trends to be described numerically as the resulting slopes are changing continuously. 

As such, quantile regression (QR) was used to define trends following the methodology in and using code provided by, Chang et al. 2023. QR calculates a linear model that seeks to minimise the residuals with a defined proportion ($\tau$) of the points above and below the fit line. For example, the scenario $\tau$ = 0.5 splits the data 50:50 above and below the line. QR has the advantage of being insensitive to outliers and can be considered analogous to a “median” trend line. This is desirable for the longer-term trends being investigated here. The 1-sigma uncertainly for the QRs were calculated via a moving block bootstrapping method, where the data are subdivided into overlapping blocks that are $n^{1/4}$ points wide (or \textasciitilde{20} points for 20 years of hourly data). These blocks are shuffled to generate a new time series, and replicated 1000 times, and the standard deviation of these replicates calculated. This uncertainty was subsequently used in the determination of the p-value, providing a metric for the significance of the trends. 

As described the QR will be a poor representation of a time series with non-linearities (hence first investigating the LOESS) as it will be using a single slope to define the whole trend across a complex time series. As urban concentration trends can change sharply (e.g. with policy intervention), the model requires freedom to represent these, therefore they were extended to use piecewise quantile regressions (PQR) with up to two break points (resulting in three trend lines). This allows the model to have some changes in the trends calculated, capturing sufficiently large-scale changes while still being able to describe a \textasciitilde{20} year time series with a small number of coefficients.

To determine what break points to use on each time series, 109 scenarios per time series defined by 1, 2 or 3 sequential QRs were created, separated by break points occurring on the 1st of January in up to two years exculding the first and last 2 years of a time series. In the cases where there were 2 break points, these were not allowed to occur within 5 years of each other. The cases with 2 or 3 regressions were combined into a single PQR, (PQR1 and PQR2 respectively). All regressions were calculated at $\tau$ = 0.05, 0.10, 0.25, 0.50, 0.75, 0.95 and 0.95.

To select ‘change points’ from the array of break points available, the model performance was evaluated via Akaike information criterion (AIC). The piecewise scenario with the minimum AIC at $\tau$ = 0.5 were selected as the change points for a given time series. This is a decision that  trades off between there being potential change points that only exist in cases where $\tau \neq$ 0.5, and retaining the ability to relate trends to one another across all quantiles. As this study aimed to explore the trends across a large number of sites, defining the change points only using $\tau$ = 0.5 is more appropriate, but future studies that reduce the number of time series could consider investigating how change points differ with $\tau$ in more detail.

Figure \ref{method_plot} demonstrates this process on a single time series. Change points could be assessed for their efficacy (i.e. is the effect of introducing a change point important, or has it only provided a trivial reduction in the AIC?) by comparing the QR and the best PQR1 and PQR2 cases, but as this does not substantively change the results presented here, this was not done and the PQR2 fits were determined to be the most appropriate in all cases.

\begin{figure}[p]
\includegraphics[width=12cm]{figures/f1_method.pdf}
\caption{Example trend determination for an O3 time series (London Bloomsbury - GB0566A). A - O3 concentration time series (black) and monthly median climatology (red). B - Anomaly time series (black, concentration minus climatology) and 4 trend options, LOESS (red), QR (blue), PQR1 (green) and PQR2 (yellow). C - Anomaly time series (black) and PQR2 across $\tau$ = 0.05, 0.10, 0.25, 0.50, 0.75, 0.95. (purple to green)}
\label{fig:method_plot}
\end{figure}


\clearpage
\section{Results and Discussion}

\begin{figure*}[t]
\includegraphics[width=12cm]{figures/f2_conc.pdf}
\caption{Median concentrations of O\textsubscript{3} (top) and NO\textsubscript{2} (bottom) across all sites in this study, separated into sites in Europe (left) and sites in the USA (right).  Shaded regions represent the 25\textsuperscript{th} and 75\textsuperscript{th} percentiles.}
\label{fig:conc_plot}
\end{figure*}

Figure \ref{fig:conc_plot} shows the annual median, 25 th and 75 th percentile concentrations of O3 and NO2 across all the urban sites used in this study to provide context for the subsequently calculated trends. Europe shows median O3 concentrations increasing from 18 ppbv to 26 ppbv between 2000 and 2019, and USA concentrations increasing, from 26 ppbv in 2000 to 30 ppbv in 2019. Both Europe and the USA show decreasing NO2 over the same period (16 ppbv to 10 ppbv in Europe, 15 ppbv to 6 ppbv in the USA). The USA does see a return to increasing NO2 concentrations after 2020, reaching 7.8 ppbv in 2022. In section \ref{sect:case_studies} evidence for this increase is discussed. 

To assess the magnitude of the changing O3 pollution problem across the USA and Europe, the daily maximum 8-hour running mean for O3 (MDA8) was calculated. MDA8 is an important metric, used to evaluate whether a particular location is in compliance with, or exceeding, internationally recognised health limits. Air quality guidelines outlined by the World Health Organisation, stipulate a MDA8 limit of 50 ppb (WHO 2005).

The dataset that results from the application of the method described in section \ref{sect:method} has several degrees of freedom to bear in mind during discussion. For a given time series (one species at one site) there are two break points that are common for all the trend lines, and as there are 7 quantiles, this results in 21 slopes, each with an associated p-value. When comparing between sites (or even species within a site) the change points can differ. This is necessary to capture the changing trends observed at urban sites but does complicate describing them. For this discussion we have grouped the sites into those from Europe and those from the United States, though details on individual European countries are available in the SI. 
To aid in summarising, in some cases the results have been grouped into time periods of a few years – in these cases if a trend changes due to a change point both trends have been counted. For example, if between 2000 and 2004 a site were to go from increasing O3 to decreasing O3, this would be described as ‘between 2000 and 2004 there was one increasing O3 trend and one decreasing O3 trend’. In the case where a site had a change point in 2003, but the trend remained increasing, this will be counted as one increasing trend. This has the benefit that no one category can count more than then number of available time series. In visualisation, both trends would be shown.

Trends have been collated into significance categories: p $\le$ 0.05 (high certainty), 0.05 $<$ p $\le$ 0.10 (medium certainty), 0.10 $<$ p $\le$ 0.33 (low certainty) and p > 0.33 (very low certainty or no evidence). For the most part. slopes where the p-value is > 0.33 are treated as ‘no trend’ regardless of their magnitude (generally we observe that as the magnitude of the trend decreases so does its significance), though sometimes are given with a direction when required by a visualisation.

\subsection{Overview of Trends} \label{sect:overview_of_trends}
\begin{figure*}[p]
\includegraphics[width=12cm]{figures/f3_eu_arrows.pdf}
\caption{Summary of trends across Europe. Angle of arrow indicates magnitude and direction, with the following orientations: vertical up - +2.5 ppb yr\textsuperscript{-1}, horizontal - 0 ppb yr\textsuperscript{-1} and vertically down - -2.5 ppb yr\textsuperscript{-1}. The few sites with absolute trends > 2.5 ppb yr\textsuperscript{-1} have been clamped to $\pm$ 2.5 ppb yr\textsuperscript{-1}. Colour indicates significance and direction, darkest colours being most significant (p < 0.05) and green the least (p > 0.33). Blues are decreasing trends and reds are increasing trends. The upper 4 plots show O3 trends, and the lower 4 show NO2 trends. Each group of 4 groups the trends into segments of the study period. If a change point occurs in a group an arrow for both trends is shown.}

\label{fig:arrow_eu}
\end{figure*}

\begin{figure*}[p]
\includegraphics[width=12cm]{figures/f4_usa_arrows.pdf}
\caption{Summary of trends across the United States of America. Angle of arrow indicates magnitude and direction, with the following orientations: vertical up - +2.5 ppb yr\textsuperscript{-1}, horizontal - 0 ppb yr\textsuperscript{-1} and vertically down - -2.5 ppb yr\textsuperscript{-1}. The few sites with absolute trends > 2.5 ppb yr\textsuperscript{-1} have been clamped to $\pm$ 2.5 ppb yr\textsuperscript{-1}. Colour indicates significance and direction, darkest colours being most significant (p < 0.05) and green the least (p > 0.33). Blues are decreasing trends and reds are increasing trends. The upper 4 plots show O3 trends, and the lower 4 show NO2 trends. Each group of 4 groups the trends into segments of the study period. If a change point occurs in a group an arrow for both trends is shown.}
\label{fig:arrow_us}
\end{figure*}

The $\tau$ = 0.5 trends of O3 and NO2 were examined and are presented on maps in figures \ref{fig:arrow_eu} and \ref{fig:arrow_us}.

In Europe, between 2000 and 2004 there were 108 trends where O3 was increasing, and 41 where it was decreasing, 32 showed no trend. By 2015-2021, 123 had an increasing trend, 32 decreasing and 41 with no trend, indicating generally increasing O3 concentrations across Europe. This is somewhat mirrored by the trends in NO2, where in 2000-2004 69 trends where increasing, 181 decreasing and 59 with no trend, changing to 29 increasing, 241 decreasing and 39 with no trend by 2015-2021. This woudl suggest a general picture that urban locations in Europe are still on the VOC limited portion of the O3 production isopleth, and decreasing NOx increases O3 production, however, it is not quite as ubiquitous as this, as shown in \ref{sect:2020_in_europe} which discusses how O3 responded to changing NO2 following restrictions in 2020. 

This contrasts with the USA, were in 2000-2004, there were 10 sites with increasing NO2, and 0 sites increasing between 2005 and 2014, but in 2015-2021 17 sites had begun increasing in NO2 again – though notably as seen from figure \ref{fig:arrow_us} these are different sites to those that were increasing at the beginning of the century. This increase in NO2 is not clearly reflected by the O3 trends with 62 increasing, 23 decreasing and 22 showing no trend in 2000 – 2004 and 44 increasing, 34 decreasing, and 25 showing no trend in 2015-2021. A decreasing number of sites  with a positive trend in O3, suggests that the majority of decreasing NO2 trends are slowing the increase in O3. The counts of sites at all values of $\tau$ can be found in tables \ref{table:europe_slope_segs} and \ref{table:usa_slope_segs}. 

For O3 in Europe, at sites with increasing trends the median slope was 0.36 ppbv yr-1, slightly lower than in the US, where they were increasing by 0.40 ppbv yr-1. For NO2, the median rate of increase was the same in Europe (0.36 ppbv yr-1), but higher at 0.71 ppbv yr-1 in the US. For decreasing trends, O3 and NO2 in Europe and the US were all similar, O3 in Europe decreasing at -0.40 ppb yr-1, (-0.36 ppbv yr-1) in the US and for NO2 -0.39 ppbv yr-1 in Europe and -0.47 in the US. This information along with the 5 th and 95 th percentile (still at $\tau$ = 0.5) are shown in table \ref{table:slope_ranges}. 




\input{tables/t1_slope_ranges.txt}

\input{tables/t2_europe_segs_11_14.txt}

\input{tables/t3_usa_segs_11_14.txt}

\clearpage
\subsubsection{Significance of Trends} \label{sect:significance_of_trends}

\begin{figure*}[htbp]
\includegraphics[width=12cm]{figures/f5_significance_bars.pdf}
\caption{Time series of trends in O3 (left) and NO2 (right) for Europe (top) and the USA (bottom). Positive trends are contained in the bar above 0 and negative trends are contained in the bar below 0. Colour indicates significance and direction, darkest colours being most significant (p < 0.05) and green the least (p > 0.33). Blues are decreasing trends and reds are increasing trends. In this case trends with p > 0.33 have been sorted by their direction.}
\label{fig:p_bar_year}
\end{figure*}

Figure \ref{fig:p_bar_year} counts the trends by significance and direction per year. This reveals the majority of trends are of high certainty. It also reinforces the observations from section \ref{sect:overview_of_trends}, where both Europe and the US see O3 increasing at number of sites, with there being proportionally more sites with a decreasing trend in the US. This additionally shows that the number of high certainty decreasing trends in O3 has been present since ~2010, joined by a slow decline in increasing trends. 

For NO2 the feature of some sites beginning to show increasing trends in the US is clear, with this pattern beginning in 2015-2016. Between 2000 and 2019 nearly all European sites moved to a trend of decreasing NO2, however, this reversed for several sites in 2020. This trend reversal can be attributed to the reduction in NO2 concentrations in Europe which were a side effect of public health measures REF Lee 2020, Grange 2021, \textcolor{red}{(also maybe Sokhi)}, and is discussed in more detail in \ref{sect:2020_in_europe}. For this reason 2000, 2001 and 2021 onward in Europe and 2000, 2001, 2020 and 2021 onwards for the USA were excluded from this year on year analysis, as it is not possible to capture changing trends in these regions. 


\subsection{Trends Across Quantiles} \label{sect:trends_across_quantiles}

To investigate how the trends in urban O3 are generally changing across quantiles in Europe and the USA, the distribution of slopes in each region was determined for each $\tau$. This section necessarily needs to discuss two sets of quantiles referring to different calculations, as such $\tau$ is used to refer to the quantile associated with the quantile regression, and other references to quantiles are associated with the analysis of the distribution of slopes within a given $\tau$.

The distributions at the start (year = 2000) and end (year = 2019) of the 20-year period were compared (Figure \ref{fig:ridgeplot}). Generally, across all values of $\tau$, the interquartile range (IQR = 75\textsuperscript{th} quantile minus 25\textsuperscript{th} quantile) of the slope distributions in O3 was greater in 2000 compared with 2019 ($\tau$ = 0.5, IQR = 0.87 and 0.58 ppbv y-1 for 2000 and 2019 respectively). This is particularly true for Europe, where a clear reduction in the higher $\tau$ O3 slopes is observed between 2000 and 2019. For $\tau$ = 0.95, a much larger proportion of the slopes were > 1.25 ppbV yr-1 in 2000 than 2019. For $\tau$ = 0.95 in the USA, there is a reduction in the number of slopes with very negative (< 1.00 ppbV yr-1) slopes between 2000 and 2019.

\begin{sidewaysfigure}
\includegraphics[height=14cm]{figures/f6_ridgelines.pdf}
\caption{Slope density for NO2 (left) and O3 (right) for Europe (cols 1 and 3) and the USA (cols 2 and 4) in 2000 (top) and 2019 (bottom) for $\tau$ = 0.05, 0.1, 0.25, 0.5, 0.75, 0.9 and 0.95}
\label{fig:ridgeplot}
\end{sidewaysfigure}


In Europe, the bulk of NO2 slopes become more negative across all $\tau$ (Figure \ref{fig:ridgeplot}). However, in the USA, despite the majority of slopes being negative in both 2000 and 2019, there is still a proportion of slopes with positive slopes in 2019, not observed in the European distribution. This is particularly true for 0.5 >= $\tau$ <= 0.9. The IQR of the slope distributions was comparable between 2000 ($\tau$ = 0.5, IQR = 0.50 ppb yr-1) and 2019 ($\tau$ = 0.5, IQR = 0.40 ppb yr-1) for Europe, comparable to the reduction in spread observed in the USA in 2019 ($\tau$ = 0.5, IQR = 0.44 ppb yr-1) compared to 2000 ($\tau$ = 0.5, IQR = 0.57 ppb yr-1).


Annual changes in $\tau$ slopes were used to evaluate how absolute urban O3 mixing ratios and trends changed across Europe and the USA between 2000 and 2021. For both Europe and the USA, the cumulative sum of the annual median of the trends across all sites was calculated for each value of $\tau$. This was followed by the subtraction of the 2000 value to get relative change since 2000. Using this analysis, Figure \ref{fig:median_slopes_per_tau_cont_name_absolute} shows the change in median O3, NO2 and Ox mixing ratios since 2000. The median of these results for Europe and the United States of America was then derived for each year and for each $\tau$. From this, Figure \ref{fig:median_slopes_per_tau_cont_name_trends} describes how the magnitude and direction of the trend line changes annually.

In both Europe and the USA, median O3 mixing ratios have generally increased between 2000 and 2021 (middle panel, Figure \ref{fig:median_slopes_per_tau_cont_name_absolute}). This is most clearly observed in Europe, where all percentiles demonstrated a consistent increase in median O3 across the two decades. Larger increases in O3 were observed in the lower $\tau$ cases (+5.4 ppb, $\tau$ = 0.05) compared to the higher $\tau$ cases (+3.5 ppb, $\tau$ = 0.95), indicating that the lowest ambient O3 levels are increasing more rapidly than higher levels. The picture is more mixed in the USA. Similarly to Europe, the largest increases in median O3 since 2000 were observed in the lower $\tau$ cases (+4.3 ppb, $\tau$ = 0.05). In the $\tau$ = 0.50 case, median O3 increased to ca. +1.8 ppb by 2011 before plateauing between 2011 and 2021, with a similar pattern also observed in the $\tau$ = 0.75 case (+1.1 ppb up to 2009, then plateauing). In both cases, an up-tick is observed from 2019, with an overall vicennial increase in O3 of +2.0 ppb ($\tau$ = 0.50) +1.2 ppb ($\tau$ = 0.75). In contrast, in the highest $\tau$ value cases, median O3 was lower in 2021 than in 2001 (-0.18 and -0.19 ppb, $\tau$ = 0.90 and 0.95 respectively). This suggests that higher ambient levels of O3 have reduced in magnitude since 2000, compared to lower ambient levels which are continuing to increase. In both Europe and the USA, median ambient NO2 mixing ratios have reduced, with the largest reductions observed in the $\tau$ = 0.95 case (-11.6 ppb and -14.6 ppb for Europe and the USA respectively). Smaller reductions were observed in the $\tau$ = 0.05 case (-4.5 and -5.4 ppb for Europe and the USA respectively).

\begin{figure*}[h!]
\includegraphics[width=12cm]{plots/fixed_median_slopes_per_tau_continent_name_absolute_change_with_mad_ribbon.png}
\caption{Absolute change in trend line-derived median NO2, O3 and Ox mixing ratios for Europe and the USA, coloured by trend line percentile. The shaded region represents the median absolute deviation.}
\label{fig:median_slopes_per_tau_cont_name_absolute}
\end{figure*}

Taking a closer look at the annual magnitude and direction of the trends allows for a closer inspection of how O3 and NO2 trends have changed between 2000 and 2021 for each $\tau$ value (Figure \ref{fig:median_slopes_per_tau_cont_name_trends}). For Europe, all median O3 trends were positive for all $\tau$ values across all years. For higher $\tau$ values ($\tau$ = 0.75, 0.90 and 0.95), the slope dropped sharply from ca. 0.54 ppbV yr-1 to ca 0.07 ppb yr-1 ($\tau$ = 0.95) between 2000 and 2004. A smaller decrease in slope was observed for the lower $\tau$ cases (from ca. 0.35 ppb y-1 to 0.18 ppb yr-1 for the $\tau$ = 0.05 case). From 2005 onward, the slope in the O3 trend generally continued to increase up to 2021. In the lower $\tau$ cases, the median slope returned to 2000-2002 levels in 2021. For the higher $\tau$ cases, the magnitude of the median slope was between ca. 0.29 - 0.46 ppb yr-1 lower in 2021 than 2000. In the USA, a larger spread in the changing median slopes was observed between different $\tau$ values. Generally, and in contrast to Europe, the median slope in O3 steadily decreased between 2000 and 2014, before plateauing between 2014 to 2021. For the lower $\tau$ values, the slope in O3 remained positive between 2000 and 2021, but reduced by ca. 0.34 ppb y-1 between 2000 and 2014 ($\tau$ = 0.05). Trends in the higher $\tau$ cases were positive at the beginning of the century, with a transition to a negative trend observed for $\tau$ = 0.75 (in 2013), and $\tau$ = 0.90 and 0.95 (both in 2007). In the $\tau$ = 0.50 case, median trends in O3 plateau around 0 ppb y-1 from 2012, indicating no overall direction of trend.

Trends in NO2 in Europe and the USA between 2000 and 2021 are also contrasting. In both Europe and the USA, median trends remain negative between 2000 and 2021, indicating NO2 is decreasing in both cases, as observed in Figure \ref{fig:median_slopes_per_tau_cont_name_absolute}. However, the direction of the trend is different in each region. In Europe, the slope of the trend is becoming increasingly more negative with time, particularly for the higher $\tau$ cases (reduction of ca. 0.72 ppb yr-1 for the $\tau$ = 0.95 case). However, in the USA the negative trends are becoming increasingly more positive (or less negative) between 2000 and 2021 (0.29 ppb y-1 higher, $\tau$ = 0.95). 

%Interestingly, in both cases, although important changes in the direction of the trend are observed for O3 and NO2, there is no clear change in trend for Ox despite a clear reduction in its absolute median mixing ratio (Figure \ref{median_slopes_per_tau_cont_name_absolute}). For all $\tau$ values, Ox trends are negative and consistent in magnitude for each $\tau$ value. The exception to this is observed in Europe between 2000 - 2003 where the Ox trend switches sharply from being positive (ca. +1.16 ppb y-1, $\tau$ = 0.95) to negative (-0.16 ppb y-1, $\tau$ = 0.95). The Ox trend then remains negative, and consistently < 0.25 ppb y-1 in most $\tau$ cases (excluding $\tau$ = 0.95) between 2004 - 2021, and including the $\tau$ = 0.95 case from 2009 onward.

\begin{figure*}[t]
\includegraphics[width=12cm]{plots/fixed_median_slopes_per_tau_continent_name.png}
\caption{Median trend line slopes (ppb y-1) in NO2, O3 and Ox for Europe and the USA, coloured by trend line $\tau$ value. The shaded region represents the median absolute deviation.}
\label{fig:median_slopes_per_tau_cont_name_trends}
\end{figure*}

\clearpage

\subsection{Spatio-temporal Distribution of Change points} \label{sect:case_studies}

% So the Flemming paper looked at 4MDA8 levels, and there were hotspots in California and southern Europe (in particular, northern Italy and Greece... how do we tie this in with our trend analysis without looking at MDA8?

The direction and magnitude of the change points in each location were investigated to identify locations where a change in direction of the trend (i.e. a negative to positive trend) occurs. In Europe, 21 sites observed a switch from a negative O3 trend line to a positive one in the first change point. The biggest switches occurred in southern Europe, in Spain and Greece ($\Delta$3.28 ppbV in gr0022a, 2004). For a larger number of sites (41), mainly in central Europe, the first change point in O3 switched from positive to negative, up to a maximum swing of 4.55 ppbV (fr31013, 2003). In contrast, a larger number of sites observed a negative to positive switch (39) in the second change point, compared to a positive to negative switch (28). Eight of the top ten biggest switches from negative to positive in the second change point occurred in Spain and Italy ($\Delta$3.99 ppbV in es0124a), whereas the biggest switch from positive to negative showed no distinct spatial pattern. This suggests that generally across Europe, more trends in O3 were switching from positive to negative in the first 5-10 years of the 21st-century, whereas more trends were switching from negative to positive between 2010-2021. In both decades, southern European sites showed the biggest changes in O3 trend from negative to positive.
Similarly to O3, more sites showed a positive to negative trend (84) in NO2 compared to negative to positive (28) for the first change point, with the largest change points again observed in Spain ($\Delta$3.25 ppbV, es1453a). A comparable number of sites showed changes in trend direction in the second change point, (43 negative to positive, and 33 positive to negative). The positive to negative trends were generally distributed around 2010-2014, whereas the vast majority of negative to positive switches occurred in 2020 (41 out of 43 sites). The cause of this switch in 2020 across Europe is discussion in further detail in section \ref{sect:2020_in_europe}.

\begin{figure}[p]
\includegraphics[width=12cm,keepaspectratio]{figures/f9_eu_mag_map.pdf}
\caption{European sites with a change in direction of slope (O3 - left, NO2 - right) of the first (rows 1, 3) second (rows 2, 4) change points. The size of the slope is relative to the magnitude of the change in slope (slopes > 2.5 ppb yr\textsuperscript{-1} have been clamped), coloured by the year of the change point. Negative to positive change points (rows 1, 2) are presented separately to positive to negative change points (rows 3, 4).}
\label{fig:eu_changepoint_map}
\end{figure}

\begin{sidewaysfigure}
\includegraphics[height=14cm]{figures/f10_us_mag_map.pdf}
\caption{US sites with a change in direction of slope (O3 - left, NO2 - right) of the first (rows 1, 3) second (rows 2, 4) change points. The size of the slope is relative to the magnitude of the change in slope (slopes > 2.5 ppb yr\textsuperscript{-1} have been clamped), coloured by the year of the change point. Negative to positive change points (rows 1, 2) are presented separately to positive to negative change points (rows 3, 4).}
\label{fig:us_changepoint_map}
\end{sidewaysfigure}

In the USA, a comparable number of directional switches in O3 trend in the first change point occurred in the positive to negative direction (21), compared to negative to positive (17). The three largest magnitude trend changes from negative to positive occurred in Florida ($\Delta$2.43 ppbv, 15292), with all of the ten highest magnitude cases in either Florida or California. Trends switching from positive to negative in the first change point scenario were smaller in magnitude, the biggest switch being in California ($\Delta$1.53 ppbv), but with some noticeable contributions from the central states. The majority of second change points occurred between 2010-2019, though it is important to note that due to a lack of data availability, it is not possible for this analysis to pick out change points from 2020 onward in the USA dataset. In total, 14 sites switched from positive to negative, with a maximum change of $\Delta$2.25 ppb (California, 2019). A comparable number of sites (12) showed trends switching from negative to positive, with a maximum change of $\Delta$2.46 ppb (Nevada, 2019). In the NO2 dataset, there is only one site showing a switch from negative to positive in the first change point ($\Delta$1.30 ppbv, 1369, 2008), compared to 13 sites showing a positive to negative switch. Of these 13 sites, the largest magnitude is in Virginia ($\Delta$2.99 ppbv, 2003). In contrast, 19 sites show a switch in NO2 trend from negative to positive after the second change point, with the biggest change in California ($\Delta$2.16 ppbv, 1206, 2019), with only one site showing a switch from positive to negative ($\Delta$0.48 ppbv, 1369, 2013).

Generally, the largest changes from negative to positive O3 trends in the USA were in California and Florida in the first half of the 20 year period, with the reversal of this trend seen across California and Florida in the second change point. 8/14 and 4/14 of the sites showing a positive to negative transition in the second change point were located in California and Florida respectively. This is important, since a previous study published in the TOAR-I special issue highlighted that 4MDA8 values between 2010-2014 were particularly high in California (Fleming et al., 2018). 

To supplement this analysis, the 4th highest daily maximum 8-hour running mean for O3 (4MDA8) was calculated using the mixing ratio data for each site and slicing the fourth highest value on an annual basis. 4MDA8 is an important metric used by the US for determining compliance with the National Ambient Air Quality Standards for Ozone. To allow for comparison with previous studies, only the 6-month warm season (April - September) was used to calculate 4MDA8 values, which is a reasonable approach in the USA and Europe since the highest O3 values are likely to occur in the Spring-Summer months.

To investigate how 4MDA8 levels may have changed since this study, the 4MDA8 was calculated from the original hourly mixing ratio O3 dataset. Figure \ref{fig:us_4mda8_map} shows how the 4MDA8 value varies between 2000, 2007, 2014 and 2021 in the USA. Across the US, 85 sites had a 4MDA8 >= 86 in 2000, compared to 48 in 2007 and 11 in 2014. In 2016 there was an up-tick to 16 sites with 4MDA8 values at this level. Of these 16 sites, 14 are located in California, compared to 30/48 Californian sites in 2000 and 24 in 2007. All 11 sites with MDA8 >= 86 were located in California. This highlights that although fewer Californian sites are exceeding 4MDA8 >= 86 ppb with time over the 2-decade period, the Californian 4MDA8 "hotspot" identified in Fleming et al., 2018 is still present. The more general picture across the USA is that 4MDA8 values are coming down with time, particularly strongly in the central and Eastern portion of the USA but less successfully in the west.

\begin{figure*}[h]
\includegraphics[width=12cm]{figures/f11_4mda8_us.pdf}
\caption{4MDA8 values in ppb for sites in the United States in 2000, 2007, 2014 and 2021}
\label{fig:us_4mda8_map}
\end{figure*}

Southern Europe, particularly Greece and Northern Italy, were also highlighted in Flemming et al., 2018 as being 4MDA8 hotspots. To identify 4MDA8 hotspots in our dataset, sites with very high 4MDA8 values (4MDA8 >= 85 ppb) were explored. Figure \ref{fig:eu_4mda8_map} shows how the 4MDA8 values are changing between 2000, 2007, 2014 and 2021 across Europe. In 2000, there were 9 sites in which 4MDA8 >= 85 ppb. This slightly increased to 11 sites in 2007 before decreasing to 6 sites in 2014. By 2021, no sites had 4MDA8 values exceeding 85 ppb. Since the measurement sites in this dataset are not evenly distributed across the USA and Europe, it is difficult to attribute high 4MDA8 values to particular states or countries, However, it was noticeable that a large proportion of the sites were 4MDA8 >= 85 ppb were in Italy and Greece. In 2000 6/9 of high 4MDA8 sites were located in Italy, and one in Greece. In 2007, 5/11 were located in Italy and 2/11 in Greece; and in 2014 5/6 sites were located in Italy. Consistent with the Fleming et al., 2018 study, the southern European 4MDA8 hotspots are apparent throughout the 20-year period. 
%Flemming et al., 2018 also observed trends going up in 4MDA8 for parts of Spain, but this is beyond the scope of our study... maybe best not to mention!

\begin{figure*}[h]
\includegraphics[width=12cm]{figures/f12_4mda8_eu.pdf}
\caption{4MDA8 values in ppb for sites in Europe in 2000, 2007, 2014 and 2021}
\label{fig:eu_4mda8_map}
\end{figure*}

\subsubsection{2020 In Europe} \label{sect:2020_in_europe}
As noted in section \ref{sect:significance_of_trends}, in 2020 in Europe several sites switched from a long term decrease in NO2 to increasing NO2 again. The public health measures enacted due to the COVID-19 pandemic had the side effect of decreasing NO2 concentrations due to reduced traffic emissions. The removal of these measures led to an increase in NO2 mixing ratios as normal activities resumed, resulting in a positive trend in NO2 concentrations up from their lowest during the restrictions. The method applied here has detected 2020 as a change point in NO2 at \textcolor{red}{many} European sites because of the large changes in NO2 during this year. There are 42 sites that had a negative trend in 2019 and a non-negative trend in 2020. The 2019 trends ranged from 1.47 to -0.04 ppbV yr\textsuperscript{-1} and the 2020 trends ranged from 0.00 to 3.74 ppbV yr\textsuperscript{-1}. There were two sites that already had non-negative trends (fr25039 and pl0048a) by 2020, but neither were strong or significant (p > 0.33). 
The effect of this on O3 concentrations at the sites is varied.  21 of the 42 sites also measured O3, though only 3 (es1038a, fr33120 and gr0031a) also had 2020 as one of the change points for the PQR. To compare the 2020 onwards trends, new PQRs for each site were selected by requiring the second change point to occur in 2020, and then selecting where the first occurred on a time series by time series basis by minimising the AIC as before. Furthermore, only sites where the resulting slope for the 2020 onwards O3 trend had p values < 0.33 were selected. One final site (gr0031a) was removed as although it meets the data coverage criteria for the main analysis, its missing data is focused around 2019 and 2020 which has a strong influence on the trends derived. This lead to 15 sites remaining for this case study. 
The resulting alternative trends for these sites are shown in figure \ref{fig:2020_in_europe}. Additionally, as the focus is on sites with both NO2 and O3, Ox is displayed. If a site is under a VOC limited regime, the trend in Ox provides an indication whether any changes in O3 are only due to changes in NO2, or other factors. Whether a change in NO2 concentrations results in an increase or decrease in O3 concentrations depends on the if a site is sensitive to locally produced O3 (which an urban site, in general, would be expected to be), and what photochemical regime (NOx or VOC limited) the site is under. For example, in a VOC limited regime, if NO2 increases, O3 decreases but Ox remains the same, the change in O3 can be attributed to the change in NO2. If NO2 and O3 increase, Ox will also increase, indicating a NOx limited regime.
14 of the sites saw increasing O3 with increasing NO2 between 2020 and 2023 - this is expected in NOx limited regimes. Due to the need for both O3 and NO2 to be measured at both sites 13 of the sites were urban background and 2 were urban traffic, as O3 is less frequently measured at traffic sites. Urban background sites are more likely to be under NOx limited regimes as they are, by design, situated further from major roads, which are still a significant source of NOx. This does not necessarily mean that urban areas in Europe are completely NOx limited and will depend on the specific location as to whether an urban traffic or urban background site is more representative of a location. At one urban traffic site (es1529a) has seen decreasing O3 with increasing NO2 over the same period. This is more indicative of a VOC limited regime, and is more similar to the overall trends seen across Europe in the 21\textsuperscript{st} century, with more increasing O3 trends in 2015-2021 than in 2000-2004. 

The sites identified by this method do not appear to have resumed their pre-2020 trends, though this should be verified when sufficient data has been collected post-2020 where change points will have more freedom to occur. The pattern of sites switching to an increasing trend in NO2 will have varied effects on urban O3 in Europe if they continue, though as background sites are designed to be more representative or residential areas, which make up a large portion of urban areas then it could be expected to see urban O3 continue to increase. 


\begin{sidewaysfigure}
    \includegraphics[height=14cm]{figures/f13_2020_in_europe.pdf}
    \caption{Sites in Europe that had decreasing NO2 in 2019 but increasing NO2 from 2020 onwards. Monthly averaged O3 (red), NO2 (orange) and Ox (green) anomaly are shown. O3 and Ox trends differ from those presented elsewhere in the manuscript as the piecewise quantile regression where the second change point has been restricted to be in 2020.}
    \label{fig:2020_in_europe}
\end{sidewaysfigure}

\subsubsection{Recent NO2 in the USA} 
In the final years of the USA NO2 dataset, an up-tick the number of positive NO2 trends can be seen. From 2015 onwards there is an increasing number of sites with this trend reaching 17 sites trends (16 high, 1 medium certainty) in 2021, up from 0 in 2014. Anomaly time series and trends of can be viewed in figure \ref{fig:no2_in_usa}. Trends have reversed from -0.98 - -0.03 ppbv yr-1 to 0.1 - 1.78 ppb yr-1. Unfortunately, only 4 of the 17 sites have concurrent measurements of O3 so it is difficult to investigate this changing trend in NO2 from the time series analysed in this study. Only one site (8169) O3 trend has its original second change point occurring alongside its NO2 increasing change point, so the same method has been applied as in \ref{sect:2020_in_europe} where the second change point has been forced to occur in the same year as the NO2 change point. Doing this reveals that sites 8169, 14386 and 15197 have reasonably significant (p = 0.01, 0.15, 0.002 respectively) O3 trends after change point 2, when this change point is the same as that in NO2. 8169 and 14386 have decreasing O3 trends (-1.70 and -0.55 ppb yr-1) with increasing NO2 and 15197 saw increasing O3 at 1.15 ppb yr-1. These data suggest that the O3 concentrations will be sensitive to this increase in NO2, but without the co-located O3 measurement it does not provide definitive information on the direction on any changes.

\begin{sidewaysfigure}
    \includegraphics[height=14cm]{figures/f14_no2_in_usa.pdf}
\caption{Sites in the USA that had increasing NO2 since 2015. Monthly averaged O3 (red), NO2 (orange) and Ox (green) anomaly are shown. O3 and Ox trends differ from those presented elsewhere in the manuscript as the piecewise quantile regression where the second change point has been restricted to match that of the NO2 at each site.}
\label{fig:no2_in_usa}
\end{sidewaysfigure}

\section{Conclusions}  %% \conclusions[modified heading if necessary]
TEXT

%% The following commands are for the statements about the availability of data sets and/or software code corresponding to the manuscript.
%% It is strongly recommended to make use of these sections in case data sets and/or software code have been part of your research the article is based on.

\codeavailability{TEXT} %% use this section when having only software code available


\dataavailability{TEXT} %% use this section when having only data sets available


\codedataavailability{TEXT} %% use this section when having data sets and software code available


\sampleavailability{TEXT} %% use this section when having geoscientific samples available


\videosupplement{TEXT} %% use this section when having video supplements available


\appendix
\section{}    %% Appendix A
\input{tables/table_test.txt}
\subsection{}     %% Appendix A1, A2, etc.


\noappendix       %% use this to mark the end of the appendix section. Otherwise the figures might be numbered incorrectly (e.g. 10 instead of 1).

%% Regarding figures and tables in appendices, the following two options are possible depending on your general handling of figures and tables in the manuscript environment:

%% Option 1: If you sorted all figures and tables into the sections of the text, please also sort the appendix figures and appendix tables into the respective appendix sections.
%% They will be correctly named automatically.

%% Option 2: If you put all figures after the reference list, please insert appendix tables and figures after the normal tables and figures.
%% To rename them correctly to A1, A2, etc., please add the following commands in front of them:

\appendixfigures  %% needs to be added in front of appendix figures

\appendixtables   %% needs to be added in front of appendix tables

%% Please add \clearpage between each table and/or figure. Further guidelines on figures and tables can be found below.



\authorcontribution{TEXT} %% this section is mandatory

\competinginterests{TEXT} %% this section is mandatory even if you declare that no competing interests are present

\disclaimer{TEXT} %% optional section

\begin{acknowledgements}
TEXT
\end{acknowledgements}




%% REFERENCES

%% The reference list is compiled as follows:

\begin{thebibliography}{}

\bibitem[AUTHOR(YEAR)]{LABEL1}
REFERENCE 1

\bibitem[AUTHOR(YEAR)]{LABEL2}
REFERENCE 2

\end{thebibliography}

%% Since the Copernicus LaTeX package includes the BibTeX style file copernicus.bst,
%% authors experienced with BibTeX only have to include the following two lines:
%%
%% \bibliographystyle{copernicus}
%% \bibliography{example.bib}
%%
%% URLs and DOIs can be entered in your BibTeX file as:
%%
%% URL = {http://www.xyz.org/~jones/idx_g.htm}
%% DOI = {10.5194/xyz}


%% LITERATURE CITATIONS
%%
%% command                        & example result
%% \citet{jones90}|               & Jones et al. (1990)
%% \citep{jones90}|               & (Jones et al., 1990)
%% \citep{jones90,jones93}|       & (Jones et al., 1990, 1993)
%% \citep[p.~32]{jones90}|        & (Jones et al., 1990, p.~32)
%% \citep[e.g.,][]{jones90}|      & (e.g., Jones et al., 1990)
%% \citep[e.g.,][p.~32]{jones90}| & (e.g., Jones et al., 1990, p.~32)
%% \citeauthor{jones90}|          & Jones et al.
%% \citeyear{jones90}|            & 1990



%% FIGURES

%% When figures and tables are placed at the end of the MS (article in one-column style), please add \clearpage
%% between bibliography and first table and/or figure as well as between each table and/or figure.

% The figure files should be labelled correctly with Arabic numerals (e.g. fig01.jpg, fig02.png).


%% ONE-COLUMN FIGURES

%%f
%\begin{figure}[t]
%\includegraphics[width=8.3cm]{FILE NAME}
%\caption{TEXT}
%\end{figure}
%
%%% TWO-COLUMN FIGURES
%
%%f
%\begin{figure*}[t]
%\includegraphics[width=12cm]{FILE NAME}
%\caption{TEXT}
%\end{figure*}
%
%
%%% TABLES
%%%
%%% The different columns must be seperated with a & command and should
%%% end with \\ to identify the column brake.
%
%%% ONE-COLUMN TABLE
%
%%t
%\begin{table}[t]
%\caption{TEXT}
%\begin{tabular}{column = lcr}
%\tophline
%
%\middlehline
%
%\bottomhline
%\end{tabular}
%\belowtable{} % Table Footnotes
%\end{table}
%
%%% TWO-COLUMN TABLE
%
%%t
%\begin{table*}[t]
%\caption{TEXT}
%\begin{tabular}{column = lcr}
%\tophline
%
%\middlehline
%
%\bottomhline
%\end{tabular}
%\belowtable{} % Table Footnotes
%\end{table*}
%
%%% LANDSCAPE TABLE
%
%%t
%\begin{sidewaystable*}[t]
%\caption{TEXT}
%\begin{tabular}{column = lcr}
%\tophline
%
%\middlehline
%
%\bottomhline
%\end{tabular}
%\belowtable{} % Table Footnotes
%\end{sidewaystable*}
%
%
%%% MATHEMATICAL EXPRESSIONS
%
%%% All papers typeset by Copernicus Publications follow the math typesetting regulations
%%% given by the IUPAC Green Book (IUPAC: Quantities, Units and Symbols in Physical Chemistry,
%%% 2nd Edn., Blackwell Science, available at: http://old.iupac.org/publications/books/gbook/green_book_2ed.pdf, 1993).
%%%
%%% Physical quantities/variables are typeset in italic font (t for time, T for Temperature)
%%% Indices which are not defined are typeset in italic font (x, y, z, a, b, c)
%%% Items/objects which are defined are typeset in roman font (Car A, Car B)
%%% Descriptions/specifications which are defined by itself are typeset in roman font (abs, rel, ref, tot, net, ice)
%%% Abbreviations from 2 letters are typeset in roman font (RH, LAI)
%%% Vectors are identified in bold italic font using \vec{x}
%%% Matrices are identified in bold roman font
%%% Multiplication signs are typeset using the LaTeX commands \times (for vector products, grids, and exponential notations) or \cdot
%%% The character * should not be applied as mutliplication sign
%
%
%%% EQUATIONS
%
%%% Single-row equation
%
%\begin{equation}
%
%\end{equation}
%
%%% Multiline equation
%
%\begin{align}
%& 3 + 5 = 8\\
%& 3 + 5 = 8\\
%& 3 + 5 = 8
%\end{align}
%
%
%%% MATRICES
%
%\begin{matrix}
%x & y & z\\
%x & y & z\\
%x & y & z\\
%\end{matrix}
%
%
%%% ALGORITHM
%
%\begin{algorithm}
%\caption{...}
%\label{a1}
%\begin{algorithmic}
%...
%\end{algorithmic}
%\end{algorithm}
%
%
%%% CHEMICAL FORMULAS AND REACTIONS
%
%%% For formulas embedded in the text, please use \chem{}
%
%%% The reaction environment creates labels including the letter R, i.e. (R1), (R2), etc.
%
%\begin{reaction}
%%% \rightarrow should be used for normal (one-way) chemical reactions
%%% \rightleftharpoons should be used for equilibria
%%% \leftrightarrow should be used for resonance structures
%\end{reaction}
%
%
%%% PHYSICAL UNITS
%%%
%%% Please use \unit{} and apply the exponential notation


\end{document}
